
% Default to the notebook output style

    


% Inherit from the specified cell style.




    
\documentclass[11pt]{article}

    
    
    \usepackage[T1]{fontenc}
    % Nicer default font (+ math font) than Computer Modern for most use cases
    \usepackage{mathpazo}

    % Basic figure setup, for now with no caption control since it's done
    % automatically by Pandoc (which extracts ![](path) syntax from Markdown).
    \usepackage{graphicx}
    % We will generate all images so they have a width \maxwidth. This means
    % that they will get their normal width if they fit onto the page, but
    % are scaled down if they would overflow the margins.
    \makeatletter
    \def\maxwidth{\ifdim\Gin@nat@width>\linewidth\linewidth
    \else\Gin@nat@width\fi}
    \makeatother
    \let\Oldincludegraphics\includegraphics
    % Set max figure width to be 80% of text width, for now hardcoded.
    \renewcommand{\includegraphics}[1]{\Oldincludegraphics[width=.8\maxwidth]{#1}}
    % Ensure that by default, figures have no caption (until we provide a
    % proper Figure object with a Caption API and a way to capture that
    % in the conversion process - todo).
    \usepackage{caption}
    \DeclareCaptionLabelFormat{nolabel}{}
    \captionsetup{labelformat=nolabel}

    \usepackage{adjustbox} % Used to constrain images to a maximum size 
    \usepackage{xcolor} % Allow colors to be defined
    \usepackage{enumerate} % Needed for markdown enumerations to work
    \usepackage{geometry} % Used to adjust the document margins
    \usepackage{amsmath} % Equations
    \usepackage{amssymb} % Equations
    \usepackage{textcomp} % defines textquotesingle
    % Hack from http://tex.stackexchange.com/a/47451/13684:
    \AtBeginDocument{%
        \def\PYZsq{\textquotesingle}% Upright quotes in Pygmentized code
    }
    \usepackage{upquote} % Upright quotes for verbatim code
    \usepackage{eurosym} % defines \euro
    \usepackage[mathletters]{ucs} % Extended unicode (utf-8) support
    \usepackage[utf8x]{inputenc} % Allow utf-8 characters in the tex document
    \usepackage{fancyvrb} % verbatim replacement that allows latex
    \usepackage{grffile} % extends the file name processing of package graphics 
                         % to support a larger range 
    % The hyperref package gives us a pdf with properly built
    % internal navigation ('pdf bookmarks' for the table of contents,
    % internal cross-reference links, web links for URLs, etc.)
    \usepackage{hyperref}
    \usepackage{longtable} % longtable support required by pandoc >1.10
    \usepackage{booktabs}  % table support for pandoc > 1.12.2
    \usepackage[inline]{enumitem} % IRkernel/repr support (it uses the enumerate* environment)
    \usepackage[normalem]{ulem} % ulem is needed to support strikethroughs (\sout)
                                % normalem makes italics be italics, not underlines
    

    
    
    % Colors for the hyperref package
    \definecolor{urlcolor}{rgb}{0,.145,.698}
    \definecolor{linkcolor}{rgb}{.71,0.21,0.01}
    \definecolor{citecolor}{rgb}{.12,.54,.11}

    % ANSI colors
    \definecolor{ansi-black}{HTML}{3E424D}
    \definecolor{ansi-black-intense}{HTML}{282C36}
    \definecolor{ansi-red}{HTML}{E75C58}
    \definecolor{ansi-red-intense}{HTML}{B22B31}
    \definecolor{ansi-green}{HTML}{00A250}
    \definecolor{ansi-green-intense}{HTML}{007427}
    \definecolor{ansi-yellow}{HTML}{DDB62B}
    \definecolor{ansi-yellow-intense}{HTML}{B27D12}
    \definecolor{ansi-blue}{HTML}{208FFB}
    \definecolor{ansi-blue-intense}{HTML}{0065CA}
    \definecolor{ansi-magenta}{HTML}{D160C4}
    \definecolor{ansi-magenta-intense}{HTML}{A03196}
    \definecolor{ansi-cyan}{HTML}{60C6C8}
    \definecolor{ansi-cyan-intense}{HTML}{258F8F}
    \definecolor{ansi-white}{HTML}{C5C1B4}
    \definecolor{ansi-white-intense}{HTML}{A1A6B2}

    % commands and environments needed by pandoc snippets
    % extracted from the output of `pandoc -s`
    \providecommand{\tightlist}{%
      \setlength{\itemsep}{0pt}\setlength{\parskip}{0pt}}
    \DefineVerbatimEnvironment{Highlighting}{Verbatim}{commandchars=\\\{\}}
    % Add ',fontsize=\small' for more characters per line
    \newenvironment{Shaded}{}{}
    \newcommand{\KeywordTok}[1]{\textcolor[rgb]{0.00,0.44,0.13}{\textbf{{#1}}}}
    \newcommand{\DataTypeTok}[1]{\textcolor[rgb]{0.56,0.13,0.00}{{#1}}}
    \newcommand{\DecValTok}[1]{\textcolor[rgb]{0.25,0.63,0.44}{{#1}}}
    \newcommand{\BaseNTok}[1]{\textcolor[rgb]{0.25,0.63,0.44}{{#1}}}
    \newcommand{\FloatTok}[1]{\textcolor[rgb]{0.25,0.63,0.44}{{#1}}}
    \newcommand{\CharTok}[1]{\textcolor[rgb]{0.25,0.44,0.63}{{#1}}}
    \newcommand{\StringTok}[1]{\textcolor[rgb]{0.25,0.44,0.63}{{#1}}}
    \newcommand{\CommentTok}[1]{\textcolor[rgb]{0.38,0.63,0.69}{\textit{{#1}}}}
    \newcommand{\OtherTok}[1]{\textcolor[rgb]{0.00,0.44,0.13}{{#1}}}
    \newcommand{\AlertTok}[1]{\textcolor[rgb]{1.00,0.00,0.00}{\textbf{{#1}}}}
    \newcommand{\FunctionTok}[1]{\textcolor[rgb]{0.02,0.16,0.49}{{#1}}}
    \newcommand{\RegionMarkerTok}[1]{{#1}}
    \newcommand{\ErrorTok}[1]{\textcolor[rgb]{1.00,0.00,0.00}{\textbf{{#1}}}}
    \newcommand{\NormalTok}[1]{{#1}}
    
    % Additional commands for more recent versions of Pandoc
    \newcommand{\ConstantTok}[1]{\textcolor[rgb]{0.53,0.00,0.00}{{#1}}}
    \newcommand{\SpecialCharTok}[1]{\textcolor[rgb]{0.25,0.44,0.63}{{#1}}}
    \newcommand{\VerbatimStringTok}[1]{\textcolor[rgb]{0.25,0.44,0.63}{{#1}}}
    \newcommand{\SpecialStringTok}[1]{\textcolor[rgb]{0.73,0.40,0.53}{{#1}}}
    \newcommand{\ImportTok}[1]{{#1}}
    \newcommand{\DocumentationTok}[1]{\textcolor[rgb]{0.73,0.13,0.13}{\textit{{#1}}}}
    \newcommand{\AnnotationTok}[1]{\textcolor[rgb]{0.38,0.63,0.69}{\textbf{\textit{{#1}}}}}
    \newcommand{\CommentVarTok}[1]{\textcolor[rgb]{0.38,0.63,0.69}{\textbf{\textit{{#1}}}}}
    \newcommand{\VariableTok}[1]{\textcolor[rgb]{0.10,0.09,0.49}{{#1}}}
    \newcommand{\ControlFlowTok}[1]{\textcolor[rgb]{0.00,0.44,0.13}{\textbf{{#1}}}}
    \newcommand{\OperatorTok}[1]{\textcolor[rgb]{0.40,0.40,0.40}{{#1}}}
    \newcommand{\BuiltInTok}[1]{{#1}}
    \newcommand{\ExtensionTok}[1]{{#1}}
    \newcommand{\PreprocessorTok}[1]{\textcolor[rgb]{0.74,0.48,0.00}{{#1}}}
    \newcommand{\AttributeTok}[1]{\textcolor[rgb]{0.49,0.56,0.16}{{#1}}}
    \newcommand{\InformationTok}[1]{\textcolor[rgb]{0.38,0.63,0.69}{\textbf{\textit{{#1}}}}}
    \newcommand{\WarningTok}[1]{\textcolor[rgb]{0.38,0.63,0.69}{\textbf{\textit{{#1}}}}}
    
    
    % Define a nice break command that doesn't care if a line doesn't already
    % exist.
    \def\br{\hspace*{\fill} \\* }
    % Math Jax compatability definitions
    \def\gt{>}
    \def\lt{<}
    % Document parameters
    \title{Bike\_Share\_Analysis}
    
    
    

    % Pygments definitions
    
\makeatletter
\def\PY@reset{\let\PY@it=\relax \let\PY@bf=\relax%
    \let\PY@ul=\relax \let\PY@tc=\relax%
    \let\PY@bc=\relax \let\PY@ff=\relax}
\def\PY@tok#1{\csname PY@tok@#1\endcsname}
\def\PY@toks#1+{\ifx\relax#1\empty\else%
    \PY@tok{#1}\expandafter\PY@toks\fi}
\def\PY@do#1{\PY@bc{\PY@tc{\PY@ul{%
    \PY@it{\PY@bf{\PY@ff{#1}}}}}}}
\def\PY#1#2{\PY@reset\PY@toks#1+\relax+\PY@do{#2}}

\expandafter\def\csname PY@tok@w\endcsname{\def\PY@tc##1{\textcolor[rgb]{0.73,0.73,0.73}{##1}}}
\expandafter\def\csname PY@tok@c\endcsname{\let\PY@it=\textit\def\PY@tc##1{\textcolor[rgb]{0.25,0.50,0.50}{##1}}}
\expandafter\def\csname PY@tok@cp\endcsname{\def\PY@tc##1{\textcolor[rgb]{0.74,0.48,0.00}{##1}}}
\expandafter\def\csname PY@tok@k\endcsname{\let\PY@bf=\textbf\def\PY@tc##1{\textcolor[rgb]{0.00,0.50,0.00}{##1}}}
\expandafter\def\csname PY@tok@kp\endcsname{\def\PY@tc##1{\textcolor[rgb]{0.00,0.50,0.00}{##1}}}
\expandafter\def\csname PY@tok@kt\endcsname{\def\PY@tc##1{\textcolor[rgb]{0.69,0.00,0.25}{##1}}}
\expandafter\def\csname PY@tok@o\endcsname{\def\PY@tc##1{\textcolor[rgb]{0.40,0.40,0.40}{##1}}}
\expandafter\def\csname PY@tok@ow\endcsname{\let\PY@bf=\textbf\def\PY@tc##1{\textcolor[rgb]{0.67,0.13,1.00}{##1}}}
\expandafter\def\csname PY@tok@nb\endcsname{\def\PY@tc##1{\textcolor[rgb]{0.00,0.50,0.00}{##1}}}
\expandafter\def\csname PY@tok@nf\endcsname{\def\PY@tc##1{\textcolor[rgb]{0.00,0.00,1.00}{##1}}}
\expandafter\def\csname PY@tok@nc\endcsname{\let\PY@bf=\textbf\def\PY@tc##1{\textcolor[rgb]{0.00,0.00,1.00}{##1}}}
\expandafter\def\csname PY@tok@nn\endcsname{\let\PY@bf=\textbf\def\PY@tc##1{\textcolor[rgb]{0.00,0.00,1.00}{##1}}}
\expandafter\def\csname PY@tok@ne\endcsname{\let\PY@bf=\textbf\def\PY@tc##1{\textcolor[rgb]{0.82,0.25,0.23}{##1}}}
\expandafter\def\csname PY@tok@nv\endcsname{\def\PY@tc##1{\textcolor[rgb]{0.10,0.09,0.49}{##1}}}
\expandafter\def\csname PY@tok@no\endcsname{\def\PY@tc##1{\textcolor[rgb]{0.53,0.00,0.00}{##1}}}
\expandafter\def\csname PY@tok@nl\endcsname{\def\PY@tc##1{\textcolor[rgb]{0.63,0.63,0.00}{##1}}}
\expandafter\def\csname PY@tok@ni\endcsname{\let\PY@bf=\textbf\def\PY@tc##1{\textcolor[rgb]{0.60,0.60,0.60}{##1}}}
\expandafter\def\csname PY@tok@na\endcsname{\def\PY@tc##1{\textcolor[rgb]{0.49,0.56,0.16}{##1}}}
\expandafter\def\csname PY@tok@nt\endcsname{\let\PY@bf=\textbf\def\PY@tc##1{\textcolor[rgb]{0.00,0.50,0.00}{##1}}}
\expandafter\def\csname PY@tok@nd\endcsname{\def\PY@tc##1{\textcolor[rgb]{0.67,0.13,1.00}{##1}}}
\expandafter\def\csname PY@tok@s\endcsname{\def\PY@tc##1{\textcolor[rgb]{0.73,0.13,0.13}{##1}}}
\expandafter\def\csname PY@tok@sd\endcsname{\let\PY@it=\textit\def\PY@tc##1{\textcolor[rgb]{0.73,0.13,0.13}{##1}}}
\expandafter\def\csname PY@tok@si\endcsname{\let\PY@bf=\textbf\def\PY@tc##1{\textcolor[rgb]{0.73,0.40,0.53}{##1}}}
\expandafter\def\csname PY@tok@se\endcsname{\let\PY@bf=\textbf\def\PY@tc##1{\textcolor[rgb]{0.73,0.40,0.13}{##1}}}
\expandafter\def\csname PY@tok@sr\endcsname{\def\PY@tc##1{\textcolor[rgb]{0.73,0.40,0.53}{##1}}}
\expandafter\def\csname PY@tok@ss\endcsname{\def\PY@tc##1{\textcolor[rgb]{0.10,0.09,0.49}{##1}}}
\expandafter\def\csname PY@tok@sx\endcsname{\def\PY@tc##1{\textcolor[rgb]{0.00,0.50,0.00}{##1}}}
\expandafter\def\csname PY@tok@m\endcsname{\def\PY@tc##1{\textcolor[rgb]{0.40,0.40,0.40}{##1}}}
\expandafter\def\csname PY@tok@gh\endcsname{\let\PY@bf=\textbf\def\PY@tc##1{\textcolor[rgb]{0.00,0.00,0.50}{##1}}}
\expandafter\def\csname PY@tok@gu\endcsname{\let\PY@bf=\textbf\def\PY@tc##1{\textcolor[rgb]{0.50,0.00,0.50}{##1}}}
\expandafter\def\csname PY@tok@gd\endcsname{\def\PY@tc##1{\textcolor[rgb]{0.63,0.00,0.00}{##1}}}
\expandafter\def\csname PY@tok@gi\endcsname{\def\PY@tc##1{\textcolor[rgb]{0.00,0.63,0.00}{##1}}}
\expandafter\def\csname PY@tok@gr\endcsname{\def\PY@tc##1{\textcolor[rgb]{1.00,0.00,0.00}{##1}}}
\expandafter\def\csname PY@tok@ge\endcsname{\let\PY@it=\textit}
\expandafter\def\csname PY@tok@gs\endcsname{\let\PY@bf=\textbf}
\expandafter\def\csname PY@tok@gp\endcsname{\let\PY@bf=\textbf\def\PY@tc##1{\textcolor[rgb]{0.00,0.00,0.50}{##1}}}
\expandafter\def\csname PY@tok@go\endcsname{\def\PY@tc##1{\textcolor[rgb]{0.53,0.53,0.53}{##1}}}
\expandafter\def\csname PY@tok@gt\endcsname{\def\PY@tc##1{\textcolor[rgb]{0.00,0.27,0.87}{##1}}}
\expandafter\def\csname PY@tok@err\endcsname{\def\PY@bc##1{\setlength{\fboxsep}{0pt}\fcolorbox[rgb]{1.00,0.00,0.00}{1,1,1}{\strut ##1}}}
\expandafter\def\csname PY@tok@kc\endcsname{\let\PY@bf=\textbf\def\PY@tc##1{\textcolor[rgb]{0.00,0.50,0.00}{##1}}}
\expandafter\def\csname PY@tok@kd\endcsname{\let\PY@bf=\textbf\def\PY@tc##1{\textcolor[rgb]{0.00,0.50,0.00}{##1}}}
\expandafter\def\csname PY@tok@kn\endcsname{\let\PY@bf=\textbf\def\PY@tc##1{\textcolor[rgb]{0.00,0.50,0.00}{##1}}}
\expandafter\def\csname PY@tok@kr\endcsname{\let\PY@bf=\textbf\def\PY@tc##1{\textcolor[rgb]{0.00,0.50,0.00}{##1}}}
\expandafter\def\csname PY@tok@bp\endcsname{\def\PY@tc##1{\textcolor[rgb]{0.00,0.50,0.00}{##1}}}
\expandafter\def\csname PY@tok@fm\endcsname{\def\PY@tc##1{\textcolor[rgb]{0.00,0.00,1.00}{##1}}}
\expandafter\def\csname PY@tok@vc\endcsname{\def\PY@tc##1{\textcolor[rgb]{0.10,0.09,0.49}{##1}}}
\expandafter\def\csname PY@tok@vg\endcsname{\def\PY@tc##1{\textcolor[rgb]{0.10,0.09,0.49}{##1}}}
\expandafter\def\csname PY@tok@vi\endcsname{\def\PY@tc##1{\textcolor[rgb]{0.10,0.09,0.49}{##1}}}
\expandafter\def\csname PY@tok@vm\endcsname{\def\PY@tc##1{\textcolor[rgb]{0.10,0.09,0.49}{##1}}}
\expandafter\def\csname PY@tok@sa\endcsname{\def\PY@tc##1{\textcolor[rgb]{0.73,0.13,0.13}{##1}}}
\expandafter\def\csname PY@tok@sb\endcsname{\def\PY@tc##1{\textcolor[rgb]{0.73,0.13,0.13}{##1}}}
\expandafter\def\csname PY@tok@sc\endcsname{\def\PY@tc##1{\textcolor[rgb]{0.73,0.13,0.13}{##1}}}
\expandafter\def\csname PY@tok@dl\endcsname{\def\PY@tc##1{\textcolor[rgb]{0.73,0.13,0.13}{##1}}}
\expandafter\def\csname PY@tok@s2\endcsname{\def\PY@tc##1{\textcolor[rgb]{0.73,0.13,0.13}{##1}}}
\expandafter\def\csname PY@tok@sh\endcsname{\def\PY@tc##1{\textcolor[rgb]{0.73,0.13,0.13}{##1}}}
\expandafter\def\csname PY@tok@s1\endcsname{\def\PY@tc##1{\textcolor[rgb]{0.73,0.13,0.13}{##1}}}
\expandafter\def\csname PY@tok@mb\endcsname{\def\PY@tc##1{\textcolor[rgb]{0.40,0.40,0.40}{##1}}}
\expandafter\def\csname PY@tok@mf\endcsname{\def\PY@tc##1{\textcolor[rgb]{0.40,0.40,0.40}{##1}}}
\expandafter\def\csname PY@tok@mh\endcsname{\def\PY@tc##1{\textcolor[rgb]{0.40,0.40,0.40}{##1}}}
\expandafter\def\csname PY@tok@mi\endcsname{\def\PY@tc##1{\textcolor[rgb]{0.40,0.40,0.40}{##1}}}
\expandafter\def\csname PY@tok@il\endcsname{\def\PY@tc##1{\textcolor[rgb]{0.40,0.40,0.40}{##1}}}
\expandafter\def\csname PY@tok@mo\endcsname{\def\PY@tc##1{\textcolor[rgb]{0.40,0.40,0.40}{##1}}}
\expandafter\def\csname PY@tok@ch\endcsname{\let\PY@it=\textit\def\PY@tc##1{\textcolor[rgb]{0.25,0.50,0.50}{##1}}}
\expandafter\def\csname PY@tok@cm\endcsname{\let\PY@it=\textit\def\PY@tc##1{\textcolor[rgb]{0.25,0.50,0.50}{##1}}}
\expandafter\def\csname PY@tok@cpf\endcsname{\let\PY@it=\textit\def\PY@tc##1{\textcolor[rgb]{0.25,0.50,0.50}{##1}}}
\expandafter\def\csname PY@tok@c1\endcsname{\let\PY@it=\textit\def\PY@tc##1{\textcolor[rgb]{0.25,0.50,0.50}{##1}}}
\expandafter\def\csname PY@tok@cs\endcsname{\let\PY@it=\textit\def\PY@tc##1{\textcolor[rgb]{0.25,0.50,0.50}{##1}}}

\def\PYZbs{\char`\\}
\def\PYZus{\char`\_}
\def\PYZob{\char`\{}
\def\PYZcb{\char`\}}
\def\PYZca{\char`\^}
\def\PYZam{\char`\&}
\def\PYZlt{\char`\<}
\def\PYZgt{\char`\>}
\def\PYZsh{\char`\#}
\def\PYZpc{\char`\%}
\def\PYZdl{\char`\$}
\def\PYZhy{\char`\-}
\def\PYZsq{\char`\'}
\def\PYZdq{\char`\"}
\def\PYZti{\char`\~}
% for compatibility with earlier versions
\def\PYZat{@}
\def\PYZlb{[}
\def\PYZrb{]}
\makeatother


    % Exact colors from NB
    \definecolor{incolor}{rgb}{0.0, 0.0, 0.5}
    \definecolor{outcolor}{rgb}{0.545, 0.0, 0.0}



    
    % Prevent overflowing lines due to hard-to-break entities
    \sloppy 
    % Setup hyperref package
    \hypersetup{
      breaklinks=true,  % so long urls are correctly broken across lines
      colorlinks=true,
      urlcolor=urlcolor,
      linkcolor=linkcolor,
      citecolor=citecolor,
      }
    % Slightly bigger margins than the latex defaults
    
    \geometry{verbose,tmargin=1in,bmargin=1in,lmargin=1in,rmargin=1in}
    
    

    \begin{document}
    
    
    \maketitle
    
    

    
    \hypertarget{us-bike-share-activity-snapshot}{%
\section{2016 US Bike Share Activity
Snapshot}\label{us-bike-share-activity-snapshot}}

\hypertarget{table-of-contents}{%
\subsection{Table of Contents}\label{table-of-contents}}

\begin{itemize}
\tightlist
\item
  Section \ref{intro}
\item
  Section \ref{pose_questions}
\item
  Section \ref{wrangling}

  \begin{itemize}
  \tightlist
  \item
    Section \ref{condensing}
  \end{itemize}
\item
  Section \ref{eda}

  \begin{itemize}
  \tightlist
  \item
    Section \ref{statistics}
  \item
    Section \ref{visualizations}
  \end{itemize}
\item
  Section \ref{eda_continued}
\item
  Section \ref{conclusions}
\end{itemize}

 \#\# Introduction

\begin{quote}
\textbf{Tip}: Quoted sections like this will provide helpful
instructions on how to navigate and use a Jupyter notebook.
\end{quote}

Over the past decade, bicycle-sharing systems have been growing in
number and popularity in cities across the world. Bicycle-sharing
systems allow users to rent bicycles for short trips, typically 30
minutes or less. Thanks to the rise in information technologies, it is
easy for a user of the system to access a dock within the system to
unlock or return bicycles. These technologies also provide a wealth of
data that can be used to explore how these bike-sharing systems are
used.

In this project, you will perform an exploratory analysis on data
provided by \href{https://www.motivateco.com/}{Motivate}, a bike-share
system provider for many major cities in the United States. You will
compare the system usage between three large cities: New York City,
Chicago, and Washington, DC. You will also see if there are any
differences within each system for those users that are registered,
regular users and those users that are short-term, casual users.

     \#\# Posing Questions

Before looking at the bike sharing data, you should start by asking
questions you might want to understand about the bike share data.
Consider, for example, if you were working for Motivate. What kinds of
information would you want to know about in order to make smarter
business decisions? If you were a user of the bike-share service, what
factors might influence how you would want to use the service?

\textbf{Question 1}: Write at least two questions related to bike
sharing that you think could be answered by data.

\textbf{Answer}: How many users are using this service? What is growth
in each month? Which city having maximum users?

\begin{quote}
\textbf{Tip}: If you double click on this cell, you will see the text
change so that all of the formatting is removed. This allows you to edit
this block of text. This block of text is written using
\href{http://daringfireball.net/projects/markdown/syntax}{Markdown},
which is a way to format text using headers, links, italics, and many
other options using a plain-text syntax. You will also use Markdown
later in the Nanodegree program. Use \textbf{Shift} + \textbf{Enter} or
\textbf{Shift} + \textbf{Return} to run the cell and show its rendered
form.
\end{quote}

     \#\# Data Collection and Wrangling

Now it's time to collect and explore our data. In this project, we will
focus on the record of individual trips taken in 2016 from our selected
cities: New York City, Chicago, and Washington, DC. Each of these cities
has a page where we can freely download the trip data.:

\begin{itemize}
\tightlist
\item
  New York City (Citi Bike):
  \href{https://www.citibikenyc.com/system-data}{Link}
\item
  Chicago (Divvy): \href{https://www.divvybikes.com/system-data}{Link}
\item
  Washington, DC (Capital Bikeshare):
  \href{https://www.capitalbikeshare.com/system-data}{Link}
\end{itemize}

If you visit these pages, you will notice that each city has a different
way of delivering its data. Chicago updates with new data twice a year,
Washington DC is quarterly, and New York City is monthly.
\textbf{However, you do not need to download the data yourself.} The
data has already been collected for you in the \texttt{/data/} folder of
the project files. While the original data for 2016 is spread among
multiple files for each city, the files in the \texttt{/data/} folder
collect all of the trip data for the year into one file per city. Some
data wrangling of inconsistencies in timestamp format within each city
has already been performed for you. In addition, a random 2\% sample of
the original data is taken to make the exploration more manageable.

\textbf{Question 2}: However, there is still a lot of data for us to
investigate, so it's a good idea to start off by looking at one entry
from each of the cities we're going to analyze. Run the first code cell
below to load some packages and functions that you'll be using in your
analysis. Then, complete the second code cell to print out the first
trip recorded from each of the cities (the second line of each data
file).

\begin{quote}
\textbf{Tip}: You can run a code cell like you formatted Markdown cells
above by clicking on the cell and using the keyboard shortcut
\textbf{Shift} + \textbf{Enter} or \textbf{Shift} + \textbf{Return}.
Alternatively, a code cell can be executed using the \textbf{Play}
button in the toolbar after selecting it. While the cell is running, you
will see an asterisk in the message to the left of the cell, i.e.
\texttt{In\ {[}*{]}:}. The asterisk will change into a number to show
that execution has completed, e.g. \texttt{In\ {[}1{]}}. If there is
output, it will show up as \texttt{Out\ {[}1{]}:}, with an appropriate
number to match the ``In'' number.
\end{quote}

    \begin{Verbatim}[commandchars=\\\{\}]
{\color{incolor}In [{\color{incolor}1}]:} \PY{c+c1}{\PYZsh{}\PYZsh{} import all necessary packages and functions.}
        \PY{k+kn}{import} \PY{n+nn}{csv} \PY{c+c1}{\PYZsh{} read and write csv files}
        \PY{k+kn}{from} \PY{n+nn}{datetime} \PY{k}{import} \PY{n}{datetime} \PY{c+c1}{\PYZsh{} operations to parse dates}
        \PY{k+kn}{from} \PY{n+nn}{pprint} \PY{k}{import} \PY{n}{pprint} \PY{c+c1}{\PYZsh{} use to print data structures like dictionaries in}
        \PY{k+kn}{import} \PY{n+nn}{time}                      \PY{c+c1}{\PYZsh{} a nicer way than the base print function.}
        \PY{k+kn}{import} \PY{n+nn}{calendar}
        
        \PY{k}{def} \PY{n+nf}{print\PYZus{}first\PYZus{}point}\PY{p}{(}\PY{n}{filename}\PY{p}{)}\PY{p}{:}
            \PY{l+s+sd}{\PYZdq{}\PYZdq{}\PYZdq{}}
        \PY{l+s+sd}{    This function prints and returns the first data point (second row) from}
        \PY{l+s+sd}{    a csv file that includes a header row.}
        \PY{l+s+sd}{    \PYZdq{}\PYZdq{}\PYZdq{}}
            \PY{c+c1}{\PYZsh{} print city name for reference}
            \PY{n}{city} \PY{o}{=} \PY{n}{filename}\PY{o}{.}\PY{n}{split}\PY{p}{(}\PY{l+s+s1}{\PYZsq{}}\PY{l+s+s1}{\PYZhy{}}\PY{l+s+s1}{\PYZsq{}}\PY{p}{)}\PY{p}{[}\PY{l+m+mi}{0}\PY{p}{]}\PY{o}{.}\PY{n}{split}\PY{p}{(}\PY{l+s+s1}{\PYZsq{}}\PY{l+s+s1}{/}\PY{l+s+s1}{\PYZsq{}}\PY{p}{)}\PY{p}{[}\PY{o}{\PYZhy{}}\PY{l+m+mi}{1}\PY{p}{]}
            \PY{n+nb}{print}\PY{p}{(}\PY{l+s+s1}{\PYZsq{}}\PY{l+s+se}{\PYZbs{}n}\PY{l+s+s1}{City: }\PY{l+s+si}{\PYZob{}\PYZcb{}}\PY{l+s+s1}{\PYZsq{}}\PY{o}{.}\PY{n}{format}\PY{p}{(}\PY{n}{city}\PY{p}{)}\PY{p}{)}
            
            \PY{k}{with} \PY{n+nb}{open}\PY{p}{(}\PY{n}{filename}\PY{p}{,} \PY{l+s+s1}{\PYZsq{}}\PY{l+s+s1}{r}\PY{l+s+s1}{\PYZsq{}}\PY{p}{)} \PY{k}{as} \PY{n}{f\PYZus{}in}\PY{p}{:}
                \PY{c+c1}{\PYZsh{}\PYZsh{} TODO: Use the csv library to set up a DictReader object. \PYZsh{}\PYZsh{}}
                \PY{c+c1}{\PYZsh{}\PYZsh{} see https://docs.python.org/3/library/csv.html           \PYZsh{}\PYZsh{}}
                \PY{n}{trip\PYZus{}reader} \PY{o}{=} \PY{n}{csv}\PY{o}{.}\PY{n}{DictReader}\PY{p}{(}\PY{n}{f\PYZus{}in}\PY{p}{)}
                
                \PY{c+c1}{\PYZsh{}\PYZsh{} TODO: Use a function on the DictReader object to read the     \PYZsh{}\PYZsh{}}
                \PY{c+c1}{\PYZsh{}\PYZsh{} first trip from the data file and store it in a variable.     \PYZsh{}\PYZsh{}}
                \PY{c+c1}{\PYZsh{}\PYZsh{} see https://docs.python.org/3/library/csv.html\PYZsh{}reader\PYZhy{}objects \PYZsh{}\PYZsh{}}
                \PY{n}{first\PYZus{}trip} \PY{o}{=} \PY{n+nb}{next}\PY{p}{(}\PY{n}{trip\PYZus{}reader}\PY{p}{)}
                
                \PY{c+c1}{\PYZsh{}\PYZsh{} TODO: Use the pprint library to print the first trip. \PYZsh{}\PYZsh{}}
                \PY{c+c1}{\PYZsh{}\PYZsh{} see https://docs.python.org/3/library/pprint.html     \PYZsh{}\PYZsh{}}
                \PY{n}{pprint}\PY{p}{(}\PY{n}{first\PYZus{}trip}\PY{p}{)}
            \PY{c+c1}{\PYZsh{} output city name and first trip for later testing}
            \PY{k}{return} \PY{p}{(}\PY{n}{city}\PY{p}{,} \PY{n}{first\PYZus{}trip}\PY{p}{)}
        
        \PY{c+c1}{\PYZsh{} list of files for each city}
        \PY{n}{data\PYZus{}files} \PY{o}{=} \PY{p}{[}\PY{l+s+s1}{\PYZsq{}}\PY{l+s+s1}{./data/NYC\PYZhy{}CitiBike\PYZhy{}2016.csv}\PY{l+s+s1}{\PYZsq{}}\PY{p}{,}
                      \PY{l+s+s1}{\PYZsq{}}\PY{l+s+s1}{./data/Chicago\PYZhy{}Divvy\PYZhy{}2016.csv}\PY{l+s+s1}{\PYZsq{}}\PY{p}{,}
                      \PY{l+s+s1}{\PYZsq{}}\PY{l+s+s1}{./data/Washington\PYZhy{}CapitalBikeshare\PYZhy{}2016.csv}\PY{l+s+s1}{\PYZsq{}}\PY{p}{,}\PY{p}{]}
        
        \PY{c+c1}{\PYZsh{} print the first trip from each file, store in dictionary}
        \PY{n}{example\PYZus{}trips} \PY{o}{=} \PY{p}{\PYZob{}}\PY{p}{\PYZcb{}}
        \PY{k}{for} \PY{n}{data\PYZus{}file} \PY{o+ow}{in} \PY{n}{data\PYZus{}files}\PY{p}{:}
            \PY{n}{city}\PY{p}{,} \PY{n}{first\PYZus{}trip} \PY{o}{=} \PY{n}{print\PYZus{}first\PYZus{}point}\PY{p}{(}\PY{n}{data\PYZus{}file}\PY{p}{)}
            \PY{n}{example\PYZus{}trips}\PY{p}{[}\PY{n}{city}\PY{p}{]} \PY{o}{=} \PY{n}{first\PYZus{}trip}
\end{Verbatim}


    \begin{Verbatim}[commandchars=\\\{\}]

City: NYC
OrderedDict([('tripduration', '839'),
             ('starttime', '1/1/2016 00:09:55'),
             ('stoptime', '1/1/2016 00:23:54'),
             ('start station id', '532'),
             ('start station name', 'S 5 Pl \& S 4 St'),
             ('start station latitude', '40.710451'),
             ('start station longitude', '-73.960876'),
             ('end station id', '401'),
             ('end station name', 'Allen St \& Rivington St'),
             ('end station latitude', '40.72019576'),
             ('end station longitude', '-73.98997825'),
             ('bikeid', '17109'),
             ('usertype', 'Customer'),
             ('birth year', ''),
             ('gender', '0')])

City: Chicago
OrderedDict([('trip\_id', '9080545'),
             ('starttime', '3/31/2016 23:30'),
             ('stoptime', '3/31/2016 23:46'),
             ('bikeid', '2295'),
             ('tripduration', '926'),
             ('from\_station\_id', '156'),
             ('from\_station\_name', 'Clark St \& Wellington Ave'),
             ('to\_station\_id', '166'),
             ('to\_station\_name', 'Ashland Ave \& Wrightwood Ave'),
             ('usertype', 'Subscriber'),
             ('gender', 'Male'),
             ('birthyear', '1990')])

City: Washington
OrderedDict([('Duration (ms)', '427387'),
             ('Start date', '3/31/2016 22:57'),
             ('End date', '3/31/2016 23:04'),
             ('Start station number', '31602'),
             ('Start station', 'Park Rd \& Holmead Pl NW'),
             ('End station number', '31207'),
             ('End station', 'Georgia Ave and Fairmont St NW'),
             ('Bike number', 'W20842'),
             ('Member Type', 'Registered')])

    \end{Verbatim}

    If everything has been filled out correctly, you should see below the
printout of each city name (which has been parsed from the data file
name) that the first trip has been parsed in the form of a dictionary.
When you set up a \texttt{DictReader} object, the first row of the data
file is normally interpreted as column names. Every other row in the
data file will use those column names as keys, as a dictionary is
generated for each row.

This will be useful since we can refer to quantities by an
easily-understandable label instead of just a numeric index. For
example, if we have a trip stored in the variable \texttt{row}, then we
would rather get the trip duration from
\texttt{row{[}\textquotesingle{}duration\textquotesingle{}{]}} instead
of \texttt{row{[}0{]}}.

 \#\#\# Condensing the Trip Data

It should also be observable from the above printout that each city
provides different information. Even where the information is the same,
the column names and formats are sometimes different. To make things as
simple as possible when we get to the actual exploration, we should trim
and clean the data. Cleaning the data makes sure that the data formats
across the cities are consistent, while trimming focuses only on the
parts of the data we are most interested in to make the exploration
easier to work with.

You will generate new data files with five values of interest for each
trip: trip duration, starting month, starting hour, day of the week, and
user type. Each of these may require additional wrangling depending on
the city:

\begin{itemize}
\tightlist
\item
  \textbf{Duration}: This has been given to us in seconds (New York,
  Chicago) or milliseconds (Washington). A more natural unit of analysis
  will be if all the trip durations are given in terms of minutes.
\item
  \textbf{Month}, \textbf{Hour}, \textbf{Day of Week}: Ridership volume
  is likely to change based on the season, time of day, and whether it
  is a weekday or weekend. Use the start time of the trip to obtain
  these values. The New York City data includes the seconds in their
  timestamps, while Washington and Chicago do not. The
  \href{https://docs.python.org/3/library/datetime.html}{\texttt{datetime}}
  package will be very useful here to make the needed conversions.
\item
  \textbf{User Type}: It is possible that users who are subscribed to a
  bike-share system will have different patterns of use compared to
  users who only have temporary passes. Washington divides its users
  into two types: `Registered' for users with annual, monthly, and other
  longer-term subscriptions, and `Casual', for users with 24-hour,
  3-day, and other short-term passes. The New York and Chicago data uses
  `Subscriber' and `Customer' for these groups, respectively. For
  consistency, you will convert the Washington labels to match the other
  two.
\end{itemize}

\textbf{Question 3a}: Complete the helper functions in the code cells
below to address each of the cleaning tasks described above.

    \begin{Verbatim}[commandchars=\\\{\}]
{\color{incolor}In [{\color{incolor}23}]:} \PY{k}{def} \PY{n+nf}{duration\PYZus{}in\PYZus{}mins}\PY{p}{(}\PY{n}{datum}\PY{p}{,} \PY{n}{city}\PY{p}{)}\PY{p}{:}
             \PY{l+s+sd}{\PYZdq{}\PYZdq{}\PYZdq{}}
         \PY{l+s+sd}{    Takes as input a dictionary containing info about a single trip (datum) and}
         \PY{l+s+sd}{    its origin city (city) and returns the trip duration in units of minutes.}
         \PY{l+s+sd}{    }
         \PY{l+s+sd}{    Remember that Washington is in terms of milliseconds while Chicago and NYC}
         \PY{l+s+sd}{    are in terms of seconds. }
         \PY{l+s+sd}{    }
         \PY{l+s+sd}{    HINT: The csv module reads in all of the data as strings, including numeric}
         \PY{l+s+sd}{    values. You will need a function to convert the strings into an appropriate}
         \PY{l+s+sd}{    numeric type when making your transformations.}
         \PY{l+s+sd}{    see https://docs.python.org/3/library/functions.html}
         \PY{l+s+sd}{    \PYZdq{}\PYZdq{}\PYZdq{}}
             
             \PY{c+c1}{\PYZsh{} YOUR CODE HERE}
             \PY{c+c1}{\PYZsh{}print(\PYZdq{}City is \PYZob{}\PYZcb{}\PYZdq{}.format(city))}
             \PY{k}{if} \PY{p}{(}\PY{n}{city} \PY{o}{==} \PY{l+s+s1}{\PYZsq{}}\PY{l+s+s1}{Washington}\PY{l+s+s1}{\PYZsq{}}\PY{p}{)}\PY{p}{:}
                 \PY{n}{millis} \PY{o}{=} \PY{n}{datum}\PY{p}{[}\PY{l+s+s1}{\PYZsq{}}\PY{l+s+s1}{Duration (ms)}\PY{l+s+s1}{\PYZsq{}}\PY{p}{]}
                 \PY{n}{minutes}\PY{o}{=}\PY{p}{(}\PY{n+nb}{int}\PY{p}{(}\PY{n}{millis}\PY{p}{)}\PY{o}{/}\PY{p}{(}\PY{l+m+mi}{1000}\PY{o}{*}\PY{l+m+mi}{60}\PY{p}{)}\PY{p}{)}\PY{o}{\PYZpc{}}\PY{k}{60}
                 \PY{n}{duration} \PY{o}{=} \PY{n}{minutes}
                 \PY{k}{return} \PY{n}{duration}
                 \PY{c+c1}{\PYZsh{}print(minutes)}
             \PY{k}{elif} \PY{p}{(}\PY{n}{city} \PY{o}{==} \PY{l+s+s1}{\PYZsq{}}\PY{l+s+s1}{NYC}\PY{l+s+s1}{\PYZsq{}}\PY{p}{)}\PY{p}{:}
                  \PY{n}{time\PYZus{}duration} \PY{o}{=} \PY{n}{datum}\PY{p}{[}\PY{l+s+s1}{\PYZsq{}}\PY{l+s+s1}{tripduration}\PY{l+s+s1}{\PYZsq{}}\PY{p}{]}
                  \PY{n}{minutes} \PY{o}{=}\PY{p}{(}\PY{n+nb}{int}\PY{p}{(}\PY{n}{time\PYZus{}duration}\PY{p}{)}\PY{o}{/}\PY{l+m+mi}{60}\PY{p}{)}\PY{o}{\PYZpc{}}\PY{k}{60}
                  \PY{n}{duration} \PY{o}{=} \PY{n}{minutes}
                  \PY{k}{return} \PY{n}{duration}
                  \PY{c+c1}{\PYZsh{}print(minutes)}
             \PY{k}{elif} \PY{p}{(}\PY{n}{city} \PY{o}{==}\PY{l+s+s1}{\PYZsq{}}\PY{l+s+s1}{Chicago}\PY{l+s+s1}{\PYZsq{}}\PY{p}{)}\PY{p}{:}
                 \PY{n}{time\PYZus{}duration} \PY{o}{=} \PY{n}{datum}\PY{p}{[}\PY{l+s+s1}{\PYZsq{}}\PY{l+s+s1}{tripduration}\PY{l+s+s1}{\PYZsq{}}\PY{p}{]}
                 \PY{n}{minutes}\PY{o}{=}\PY{p}{(}\PY{n+nb}{int}\PY{p}{(}\PY{n}{time\PYZus{}duration}\PY{p}{)}\PY{o}{/}\PY{l+m+mi}{60}\PY{p}{)}\PY{o}{\PYZpc{}}\PY{k}{60}
                 \PY{n}{duration} \PY{o}{=} \PY{n}{minutes}
                 \PY{c+c1}{\PYZsh{}print(minutes)}
                 \PY{k}{return} \PY{n}{duration}
             \PY{k}{else}\PY{p}{:}
                 \PY{n+nb}{print}\PY{p}{(}\PY{l+s+s2}{\PYZdq{}}\PY{l+s+s2}{something wrong}\PY{l+s+s2}{\PYZdq{}}\PY{p}{)}
                 \PY{c+c1}{\PYZsh{}duration = False}
                 
         
         \PY{c+c1}{\PYZsh{} Some tests to check that your code works. There should be no output if all of}
         \PY{c+c1}{\PYZsh{} the assertions pass. The `example\PYZus{}trips` dictionary was obtained from when}
         \PY{c+c1}{\PYZsh{} you printed the first trip from each of the original data files.}
         \PY{n}{tests} \PY{o}{=} \PY{p}{\PYZob{}}\PY{l+s+s1}{\PYZsq{}}\PY{l+s+s1}{NYC}\PY{l+s+s1}{\PYZsq{}}\PY{p}{:} \PY{l+m+mf}{13.9833}\PY{p}{,}
                  \PY{l+s+s1}{\PYZsq{}}\PY{l+s+s1}{Chicago}\PY{l+s+s1}{\PYZsq{}}\PY{p}{:} \PY{l+m+mf}{15.4333}\PY{p}{,}
                  \PY{l+s+s1}{\PYZsq{}}\PY{l+s+s1}{Washington}\PY{l+s+s1}{\PYZsq{}}\PY{p}{:} \PY{l+m+mf}{7.1231}\PY{p}{\PYZcb{}}
         
         \PY{k}{for} \PY{n}{city} \PY{o+ow}{in} \PY{n}{tests}\PY{p}{:}
             \PY{c+c1}{\PYZsh{}print(abs(duration\PYZus{}in\PYZus{}mins(example\PYZus{}trips[city], city)))}
             \PY{k}{assert} \PY{n+nb}{abs}\PY{p}{(}\PY{n}{duration\PYZus{}in\PYZus{}mins}\PY{p}{(}\PY{n}{example\PYZus{}trips}\PY{p}{[}\PY{n}{city}\PY{p}{]}\PY{p}{,} \PY{n}{city}\PY{p}{)} \PY{o}{\PYZhy{}} \PY{n}{tests}\PY{p}{[}\PY{n}{city}\PY{p}{]}\PY{p}{)} \PY{o}{\PYZlt{}} \PY{o}{.}\PY{l+m+mi}{001}
\end{Verbatim}


    \begin{Verbatim}[commandchars=\\\{\}]
{\color{incolor}In [{\color{incolor}24}]:} \PY{k}{def} \PY{n+nf}{time\PYZus{}of\PYZus{}trip}\PY{p}{(}\PY{n}{datum}\PY{p}{,} \PY{n}{city}\PY{p}{)}\PY{p}{:}
             \PY{l+s+sd}{\PYZdq{}\PYZdq{}\PYZdq{}}
         \PY{l+s+sd}{    Takes as input a dictionary containing info about a single trip (datum) and}
         \PY{l+s+sd}{    its origin city (city) and returns the month, hour, and day of the week in}
         \PY{l+s+sd}{    which the trip was made.}
         \PY{l+s+sd}{    }
         \PY{l+s+sd}{    Remember that NYC includes seconds, while Washington and Chicago do not.}
         \PY{l+s+sd}{    }
         \PY{l+s+sd}{    HINT: You should use the datetime module to parse the original date}
         \PY{l+s+sd}{    strings into a format that is useful for extracting the desired information.}
         \PY{l+s+sd}{    see https://docs.python.org/3/library/datetime.html\PYZsh{}strftime\PYZhy{}and\PYZhy{}strptime\PYZhy{}behavior}
         \PY{l+s+sd}{    \PYZdq{}\PYZdq{}\PYZdq{}}
             
             \PY{c+c1}{\PYZsh{} YOUR CODE HERE}
             \PY{n}{fmt\PYZus{}NYC} \PY{o}{=} \PY{l+s+s1}{\PYZsq{}}\PY{l+s+s1}{\PYZpc{}}\PY{l+s+s1}{m/}\PY{l+s+si}{\PYZpc{}d}\PY{l+s+s1}{/}\PY{l+s+s1}{\PYZpc{}}\PY{l+s+s1}{Y }\PY{l+s+s1}{\PYZpc{}}\PY{l+s+s1}{H:}\PY{l+s+s1}{\PYZpc{}}\PY{l+s+s1}{M:}\PY{l+s+s1}{\PYZpc{}}\PY{l+s+s1}{S}\PY{l+s+s1}{\PYZsq{}}
             \PY{n}{fmt\PYZus{}Chicago} \PY{o}{=} \PY{l+s+s1}{\PYZsq{}}\PY{l+s+s1}{\PYZpc{}}\PY{l+s+s1}{m/}\PY{l+s+si}{\PYZpc{}d}\PY{l+s+s1}{/}\PY{l+s+s1}{\PYZpc{}}\PY{l+s+s1}{Y }\PY{l+s+s1}{\PYZpc{}}\PY{l+s+s1}{H:}\PY{l+s+s1}{\PYZpc{}}\PY{l+s+s1}{M}\PY{l+s+s1}{\PYZsq{}}
             \PY{c+c1}{\PYZsh{}print(\PYZdq{}City is \PYZob{}\PYZcb{}\PYZdq{}.format(city))}
             \PY{k}{if} \PY{p}{(}\PY{n}{city} \PY{o}{==} \PY{l+s+s1}{\PYZsq{}}\PY{l+s+s1}{Washington}\PY{l+s+s1}{\PYZsq{}}\PY{p}{)}\PY{p}{:}
                 \PY{n}{starttime} \PY{o}{=} \PY{n}{datum}\PY{p}{[}\PY{l+s+s1}{\PYZsq{}}\PY{l+s+s1}{Start date}\PY{l+s+s1}{\PYZsq{}}\PY{p}{]}
                 \PY{n}{start}\PY{o}{=} \PY{n}{datetime}\PY{o}{.}\PY{n}{strptime}\PY{p}{(}\PY{n}{starttime}\PY{p}{,} \PY{n}{fmt\PYZus{}Chicago}\PY{p}{)}
                 \PY{n}{month} \PY{o}{=} \PY{n}{start}\PY{o}{.}\PY{n}{month}
                 \PY{c+c1}{\PYZsh{}month = format(month\PYZus{},\PYZsq{}02d\PYZsq{})}
                 \PY{n}{hour} \PY{o}{=} \PY{n}{start}\PY{o}{.}\PY{n}{hour}
                 \PY{c+c1}{\PYZsh{}hour = format(hour\PYZus{},\PYZsq{}02d\PYZsq{})}
                 \PY{n}{day\PYZus{}of\PYZus{}week} \PY{o}{=} \PY{n}{calendar}\PY{o}{.}\PY{n}{day\PYZus{}name}\PY{p}{[}\PY{n}{start}\PY{o}{.}\PY{n}{weekday}\PY{p}{(}\PY{p}{)}\PY{p}{]}
         
             \PY{k}{elif} \PY{p}{(}\PY{n}{city} \PY{o}{==}\PY{l+s+s1}{\PYZsq{}}\PY{l+s+s1}{NYC}\PY{l+s+s1}{\PYZsq{}}\PY{p}{)}\PY{p}{:}
                 \PY{n}{starttime} \PY{o}{=} \PY{n}{datum}\PY{p}{[}\PY{l+s+s1}{\PYZsq{}}\PY{l+s+s1}{starttime}\PY{l+s+s1}{\PYZsq{}}\PY{p}{]}
                 \PY{n}{start}\PY{o}{=} \PY{n}{datetime}\PY{o}{.}\PY{n}{strptime}\PY{p}{(}\PY{n}{starttime}\PY{p}{,} \PY{n}{fmt\PYZus{}NYC}\PY{p}{)}
                 \PY{n}{month} \PY{o}{=} \PY{n}{start}\PY{o}{.}\PY{n}{month}
                 \PY{c+c1}{\PYZsh{}month = format(month\PYZus{},\PYZsq{}02d\PYZsq{})}
                 \PY{n}{hour} \PY{o}{=} \PY{n}{start}\PY{o}{.}\PY{n}{hour}
                 \PY{c+c1}{\PYZsh{}hour = format(hour\PYZus{},\PYZsq{}02d\PYZsq{})}
                 \PY{n}{day\PYZus{}of\PYZus{}week} \PY{o}{=} \PY{n}{calendar}\PY{o}{.}\PY{n}{day\PYZus{}name}\PY{p}{[}\PY{n}{start}\PY{o}{.}\PY{n}{weekday}\PY{p}{(}\PY{p}{)}\PY{p}{]}
         
             \PY{k}{elif} \PY{p}{(}\PY{n}{city} \PY{o}{==}\PY{l+s+s1}{\PYZsq{}}\PY{l+s+s1}{Chicago}\PY{l+s+s1}{\PYZsq{}}\PY{p}{)}\PY{p}{:}
                 \PY{n}{starttime} \PY{o}{=} \PY{n}{datum}\PY{p}{[}\PY{l+s+s1}{\PYZsq{}}\PY{l+s+s1}{starttime}\PY{l+s+s1}{\PYZsq{}}\PY{p}{]}
                 \PY{n}{start}\PY{o}{=} \PY{n}{datetime}\PY{o}{.}\PY{n}{strptime}\PY{p}{(}\PY{n}{starttime}\PY{p}{,} \PY{n}{fmt\PYZus{}Chicago}\PY{p}{)}
                 \PY{n}{month} \PY{o}{=} \PY{n}{start}\PY{o}{.}\PY{n}{month}
                 \PY{c+c1}{\PYZsh{}month = format(month\PYZus{},\PYZsq{}02d\PYZsq{})}
                 \PY{n}{hour} \PY{o}{=} \PY{n}{start}\PY{o}{.}\PY{n}{hour}
                 \PY{c+c1}{\PYZsh{}hour = format(hour\PYZus{},\PYZsq{}02d\PYZsq{})}
                 \PY{n}{day\PYZus{}of\PYZus{}week} \PY{o}{=} \PY{n}{calendar}\PY{o}{.}\PY{n}{day\PYZus{}name}\PY{p}{[}\PY{n}{start}\PY{o}{.}\PY{n}{weekday}\PY{p}{(}\PY{p}{)}\PY{p}{]}
         
             \PY{k}{else}\PY{p}{:}
                 \PY{n+nb}{print}\PY{p}{(}\PY{l+s+s2}{\PYZdq{}}\PY{l+s+s2}{Something is wrong. Please check input}\PY{l+s+s2}{\PYZdq{}}\PY{p}{)}
                 \PY{k}{return} \PY{k+kc}{False}
             
             \PY{k}{return} \PY{p}{(}\PY{n}{month}\PY{p}{,} \PY{n}{hour}\PY{p}{,} \PY{n}{day\PYZus{}of\PYZus{}week}\PY{p}{)}
         
         
         \PY{c+c1}{\PYZsh{} Some tests to check that your code works. There should be no output if all of}
         \PY{c+c1}{\PYZsh{} the assertions pass. The `example\PYZus{}trips` dictionary was obtained from when}
         \PY{c+c1}{\PYZsh{} you printed the first trip from each of the original data files.}
         \PY{n}{tests} \PY{o}{=} \PY{p}{\PYZob{}}\PY{l+s+s1}{\PYZsq{}}\PY{l+s+s1}{NYC}\PY{l+s+s1}{\PYZsq{}}\PY{p}{:} \PY{p}{(}\PY{l+m+mi}{1}\PY{p}{,} \PY{l+m+mi}{0}\PY{p}{,} \PY{l+s+s1}{\PYZsq{}}\PY{l+s+s1}{Friday}\PY{l+s+s1}{\PYZsq{}}\PY{p}{)}\PY{p}{,}
                  \PY{l+s+s1}{\PYZsq{}}\PY{l+s+s1}{Chicago}\PY{l+s+s1}{\PYZsq{}}\PY{p}{:} \PY{p}{(}\PY{l+m+mi}{3}\PY{p}{,} \PY{l+m+mi}{23}\PY{p}{,} \PY{l+s+s1}{\PYZsq{}}\PY{l+s+s1}{Thursday}\PY{l+s+s1}{\PYZsq{}}\PY{p}{)}\PY{p}{,}
                  \PY{l+s+s1}{\PYZsq{}}\PY{l+s+s1}{Washington}\PY{l+s+s1}{\PYZsq{}}\PY{p}{:} \PY{p}{(}\PY{l+m+mi}{3}\PY{p}{,} \PY{l+m+mi}{22}\PY{p}{,} \PY{l+s+s1}{\PYZsq{}}\PY{l+s+s1}{Thursday}\PY{l+s+s1}{\PYZsq{}}\PY{p}{)}\PY{p}{\PYZcb{}}
         
         \PY{k}{for} \PY{n}{city} \PY{o+ow}{in} \PY{n}{tests}\PY{p}{:}
             \PY{c+c1}{\PYZsh{}print(time\PYZus{}of\PYZus{}trip(example\PYZus{}trips[city], city))}
             \PY{k}{assert} \PY{n}{time\PYZus{}of\PYZus{}trip}\PY{p}{(}\PY{n}{example\PYZus{}trips}\PY{p}{[}\PY{n}{city}\PY{p}{]}\PY{p}{,} \PY{n}{city}\PY{p}{)} \PY{o}{==} \PY{n}{tests}\PY{p}{[}\PY{n}{city}\PY{p}{]}
\end{Verbatim}


    \begin{Verbatim}[commandchars=\\\{\}]
{\color{incolor}In [{\color{incolor}25}]:} \PY{k}{def} \PY{n+nf}{type\PYZus{}of\PYZus{}user}\PY{p}{(}\PY{n}{datum}\PY{p}{,} \PY{n}{city}\PY{p}{)}\PY{p}{:}
             \PY{l+s+sd}{\PYZdq{}\PYZdq{}\PYZdq{}}
         \PY{l+s+sd}{    Takes as input a dictionary containing info about a single trip (datum) and}
         \PY{l+s+sd}{    its origin city (city) and returns the type of system user that made the}
         \PY{l+s+sd}{    trip.}
         \PY{l+s+sd}{    }
         \PY{l+s+sd}{    Remember that Washington has different category names compared to Chicago}
         \PY{l+s+sd}{    and NYC. }
         \PY{l+s+sd}{    \PYZdq{}\PYZdq{}\PYZdq{}}
             
             \PY{c+c1}{\PYZsh{} YOUR CODE HERE}
             \PY{k}{if} \PY{p}{(}\PY{n}{city} \PY{o}{==} \PY{l+s+s1}{\PYZsq{}}\PY{l+s+s1}{Washington}\PY{l+s+s1}{\PYZsq{}}\PY{p}{)}\PY{p}{:}
                 \PY{n}{user\PYZus{}type} \PY{o}{=} \PY{n}{datum}\PY{p}{[}\PY{l+s+s1}{\PYZsq{}}\PY{l+s+s1}{Member Type}\PY{l+s+s1}{\PYZsq{}}\PY{p}{]}
                 \PY{k}{if}\PY{p}{(}\PY{n}{user\PYZus{}type} \PY{o}{==} \PY{l+s+s1}{\PYZsq{}}\PY{l+s+s1}{Registered}\PY{l+s+s1}{\PYZsq{}}\PY{p}{)}\PY{p}{:}
                     \PY{n}{user\PYZus{}type} \PY{o}{=} \PY{l+s+s1}{\PYZsq{}}\PY{l+s+s1}{Subscriber}\PY{l+s+s1}{\PYZsq{}}
                 \PY{k}{if}\PY{p}{(}\PY{n}{user\PYZus{}type} \PY{o}{==} \PY{l+s+s1}{\PYZsq{}}\PY{l+s+s1}{Casual}\PY{l+s+s1}{\PYZsq{}}\PY{p}{)}\PY{p}{:}
                     \PY{n}{user\PYZus{}type} \PY{o}{=} \PY{l+s+s1}{\PYZsq{}}\PY{l+s+s1}{Customer}\PY{l+s+s1}{\PYZsq{}}
             \PY{k}{elif} \PY{p}{(}\PY{n}{city} \PY{o}{==}\PY{l+s+s1}{\PYZsq{}}\PY{l+s+s1}{NYC}\PY{l+s+s1}{\PYZsq{}}\PY{p}{)}\PY{p}{:}
                 \PY{n}{user\PYZus{}type} \PY{o}{=} \PY{n}{datum}\PY{p}{[}\PY{l+s+s1}{\PYZsq{}}\PY{l+s+s1}{usertype}\PY{l+s+s1}{\PYZsq{}}\PY{p}{]}
         
             \PY{k}{elif} \PY{p}{(}\PY{n}{city} \PY{o}{==}\PY{l+s+s1}{\PYZsq{}}\PY{l+s+s1}{Chicago}\PY{l+s+s1}{\PYZsq{}}\PY{p}{)}\PY{p}{:}
                 \PY{n}{user\PYZus{}type} \PY{o}{=} \PY{n}{datum}\PY{p}{[}\PY{l+s+s1}{\PYZsq{}}\PY{l+s+s1}{usertype}\PY{l+s+s1}{\PYZsq{}}\PY{p}{]}
         
             \PY{k}{else}\PY{p}{:}
                 \PY{n+nb}{print}\PY{p}{(}\PY{l+s+s2}{\PYZdq{}}\PY{l+s+s2}{Something is wrong. Please check input}\PY{l+s+s2}{\PYZdq{}}\PY{p}{)}
                 \PY{k}{return} \PY{k+kc}{False}
             
             \PY{k}{return} \PY{n}{user\PYZus{}type}
         
         
         \PY{c+c1}{\PYZsh{} Some tests to check that your code works. There should be no output if all of}
         \PY{c+c1}{\PYZsh{} the assertions pass. The `example\PYZus{}trips` dictionary was obtained from when}
         \PY{c+c1}{\PYZsh{} you printed the first trip from each of the original data files.}
         \PY{n}{tests} \PY{o}{=} \PY{p}{\PYZob{}}\PY{l+s+s1}{\PYZsq{}}\PY{l+s+s1}{NYC}\PY{l+s+s1}{\PYZsq{}}\PY{p}{:} \PY{l+s+s1}{\PYZsq{}}\PY{l+s+s1}{Customer}\PY{l+s+s1}{\PYZsq{}}\PY{p}{,}
                  \PY{l+s+s1}{\PYZsq{}}\PY{l+s+s1}{Chicago}\PY{l+s+s1}{\PYZsq{}}\PY{p}{:} \PY{l+s+s1}{\PYZsq{}}\PY{l+s+s1}{Subscriber}\PY{l+s+s1}{\PYZsq{}}\PY{p}{,}
                  \PY{l+s+s1}{\PYZsq{}}\PY{l+s+s1}{Washington}\PY{l+s+s1}{\PYZsq{}}\PY{p}{:} \PY{l+s+s1}{\PYZsq{}}\PY{l+s+s1}{Subscriber}\PY{l+s+s1}{\PYZsq{}}\PY{p}{\PYZcb{}}
         
         \PY{k}{for} \PY{n}{city} \PY{o+ow}{in} \PY{n}{tests}\PY{p}{:}
             \PY{c+c1}{\PYZsh{}print(type\PYZus{}of\PYZus{}user(example\PYZus{}trips[city], city))}
             \PY{k}{assert} \PY{n}{type\PYZus{}of\PYZus{}user}\PY{p}{(}\PY{n}{example\PYZus{}trips}\PY{p}{[}\PY{n}{city}\PY{p}{]}\PY{p}{,} \PY{n}{city}\PY{p}{)} \PY{o}{==} \PY{n}{tests}\PY{p}{[}\PY{n}{city}\PY{p}{]}
\end{Verbatim}


    \textbf{Question 3b}: Now, use the helper functions you wrote above to
create a condensed data file for each city consisting only of the data
fields indicated above. In the \texttt{/examples/} folder, you will see
an example datafile from the
\href{http://www.bayareabikeshare.com/open-data}{Bay Area Bike Share}
before and after conversion. Make sure that your output is formatted to
be consistent with the example file.

    \begin{Verbatim}[commandchars=\\\{\}]
{\color{incolor}In [{\color{incolor}26}]:} \PY{k}{def} \PY{n+nf}{condense\PYZus{}data}\PY{p}{(}\PY{n}{in\PYZus{}file}\PY{p}{,} \PY{n}{out\PYZus{}file}\PY{p}{,} \PY{n}{city}\PY{p}{)}\PY{p}{:}
             \PY{l+s+sd}{\PYZdq{}\PYZdq{}\PYZdq{}}
         \PY{l+s+sd}{    This function takes full data from the specified input file}
         \PY{l+s+sd}{    and writes the condensed data to a specified output file. The city}
         \PY{l+s+sd}{    argument determines how the input file will be parsed.}
         \PY{l+s+sd}{    }
         \PY{l+s+sd}{    HINT: See the cell below to see how the arguments are structured!}
         \PY{l+s+sd}{    \PYZdq{}\PYZdq{}\PYZdq{}}
             
             \PY{k}{with} \PY{n+nb}{open}\PY{p}{(}\PY{n}{out\PYZus{}file}\PY{p}{,} \PY{l+s+s1}{\PYZsq{}}\PY{l+s+s1}{w}\PY{l+s+s1}{\PYZsq{}}\PY{p}{,} \PY{n}{newline}\PY{o}{=}\PY{l+s+s1}{\PYZsq{}}\PY{l+s+s1}{\PYZsq{}}\PY{p}{)} \PY{k}{as} \PY{n}{f\PYZus{}out}\PY{p}{,} \PY{n+nb}{open}\PY{p}{(}\PY{n}{in\PYZus{}file}\PY{p}{,} \PY{l+s+s1}{\PYZsq{}}\PY{l+s+s1}{r}\PY{l+s+s1}{\PYZsq{}}\PY{p}{)} \PY{k}{as} \PY{n}{f\PYZus{}in}\PY{p}{:}
                 \PY{c+c1}{\PYZsh{} set up csv DictWriter object \PYZhy{} writer requires column names for the}
                 \PY{c+c1}{\PYZsh{} first row as the \PYZdq{}fieldnames\PYZdq{} argument}
                 \PY{n}{out\PYZus{}colnames} \PY{o}{=} \PY{p}{[}\PY{l+s+s1}{\PYZsq{}}\PY{l+s+s1}{duration}\PY{l+s+s1}{\PYZsq{}}\PY{p}{,} \PY{l+s+s1}{\PYZsq{}}\PY{l+s+s1}{month}\PY{l+s+s1}{\PYZsq{}}\PY{p}{,} \PY{l+s+s1}{\PYZsq{}}\PY{l+s+s1}{hour}\PY{l+s+s1}{\PYZsq{}}\PY{p}{,} \PY{l+s+s1}{\PYZsq{}}\PY{l+s+s1}{day\PYZus{}of\PYZus{}week}\PY{l+s+s1}{\PYZsq{}}\PY{p}{,} \PY{l+s+s1}{\PYZsq{}}\PY{l+s+s1}{user\PYZus{}type}\PY{l+s+s1}{\PYZsq{}}\PY{p}{]}        
                 \PY{n}{trip\PYZus{}writer} \PY{o}{=} \PY{n}{csv}\PY{o}{.}\PY{n}{DictWriter}\PY{p}{(}\PY{n}{f\PYZus{}out}\PY{p}{,} \PY{n}{fieldnames} \PY{o}{=} \PY{n}{out\PYZus{}colnames}\PY{p}{)}
                 \PY{n}{trip\PYZus{}writer}\PY{o}{.}\PY{n}{writeheader}\PY{p}{(}\PY{p}{)}
                 
                 \PY{c+c1}{\PYZsh{}\PYZsh{} TODO: set up csv DictReader object \PYZsh{}\PYZsh{}}
                 \PY{n}{trip\PYZus{}reader} \PY{o}{=} \PY{n}{csv}\PY{o}{.}\PY{n}{DictReader}\PY{p}{(}\PY{n}{f\PYZus{}in}\PY{p}{)}
         
                 \PY{c+c1}{\PYZsh{} collect data from and process each row}
                 \PY{k}{for} \PY{n}{row} \PY{o+ow}{in} \PY{n}{trip\PYZus{}reader}\PY{p}{:}
                     \PY{c+c1}{\PYZsh{} set up a dictionary to hold the values for the cleaned and trimmed}
                     \PY{c+c1}{\PYZsh{} data point}
                     \PY{n}{new\PYZus{}point} \PY{o}{=} \PY{p}{\PYZob{}}\PY{p}{\PYZcb{}}
         
                     \PY{c+c1}{\PYZsh{}\PYZsh{} TODO: use the helper functions to get the cleaned data from  \PYZsh{}\PYZsh{}}
                     \PY{c+c1}{\PYZsh{}\PYZsh{} the original data dictionaries.                              \PYZsh{}\PYZsh{}}
                     \PY{c+c1}{\PYZsh{}\PYZsh{} Note that the keys for the new\PYZus{}point dictionary should match \PYZsh{}\PYZsh{}}
                     \PY{c+c1}{\PYZsh{}\PYZsh{} the column names set in the DictWriter object above.         \PYZsh{}\PYZsh{}}
                     \PY{n}{duration1} \PY{o}{=} \PY{n}{duration\PYZus{}in\PYZus{}mins}\PY{p}{(}\PY{n}{row}\PY{p}{,} \PY{n}{city}\PY{p}{)}
                     \PY{n}{month1}\PY{p}{,} \PY{n}{hours1}\PY{p}{,} \PY{n}{day\PYZus{}of\PYZus{}week1} \PY{o}{=} \PY{n}{time\PYZus{}of\PYZus{}trip}\PY{p}{(}\PY{n}{row}\PY{p}{,} \PY{n}{city}\PY{p}{)}
                     \PY{n}{user\PYZus{}type1} \PY{o}{=} \PY{n}{type\PYZus{}of\PYZus{}user}\PY{p}{(}\PY{n}{row}\PY{p}{,} \PY{n}{city}\PY{p}{)}
                     \PY{c+c1}{\PYZsh{}\PYZsh{} TODO: write the processed information to the output file.     \PYZsh{}\PYZsh{}}
                     \PY{c+c1}{\PYZsh{}\PYZsh{} see https://docs.python.org/3/library/csv.html\PYZsh{}writer\PYZhy{}objects \PYZsh{}\PYZsh{}}
                     \PY{n}{new\PYZus{}point} \PY{o}{=} \PY{p}{\PYZob{}}\PY{l+s+s1}{\PYZsq{}}\PY{l+s+s1}{duration}\PY{l+s+s1}{\PYZsq{}}\PY{p}{:} \PY{n}{duration1} \PY{p}{,} \PY{l+s+s1}{\PYZsq{}}\PY{l+s+s1}{month}\PY{l+s+s1}{\PYZsq{}} \PY{p}{:} \PY{n}{month1}\PY{p}{,} \PY{l+s+s1}{\PYZsq{}}\PY{l+s+s1}{hour}\PY{l+s+s1}{\PYZsq{}} \PY{p}{:} \PY{n}{hours1} \PY{p}{,} \PY{l+s+s1}{\PYZsq{}}\PY{l+s+s1}{day\PYZus{}of\PYZus{}week}\PY{l+s+s1}{\PYZsq{}} \PY{p}{:} \PY{n}{day\PYZus{}of\PYZus{}week1}\PY{p}{,} \PY{l+s+s1}{\PYZsq{}}\PY{l+s+s1}{user\PYZus{}type}\PY{l+s+s1}{\PYZsq{}}\PY{p}{:} \PY{n}{user\PYZus{}type1}\PY{p}{\PYZcb{}}
                     \PY{n}{trip\PYZus{}writer}\PY{o}{.}\PY{n}{writerow}\PY{p}{(}\PY{n}{new\PYZus{}point}\PY{p}{)}
                     
\end{Verbatim}


    \begin{Verbatim}[commandchars=\\\{\}]
{\color{incolor}In [{\color{incolor}27}]:} \PY{c+c1}{\PYZsh{} Run this cell to check your work}
         \PY{n}{city\PYZus{}info} \PY{o}{=} \PY{p}{\PYZob{}}\PY{l+s+s1}{\PYZsq{}}\PY{l+s+s1}{Washington}\PY{l+s+s1}{\PYZsq{}}\PY{p}{:} \PY{p}{\PYZob{}}\PY{l+s+s1}{\PYZsq{}}\PY{l+s+s1}{in\PYZus{}file}\PY{l+s+s1}{\PYZsq{}}\PY{p}{:} \PY{l+s+s1}{\PYZsq{}}\PY{l+s+s1}{./data/Washington\PYZhy{}CapitalBikeshare\PYZhy{}2016.csv}\PY{l+s+s1}{\PYZsq{}}\PY{p}{,}
                                     \PY{l+s+s1}{\PYZsq{}}\PY{l+s+s1}{out\PYZus{}file}\PY{l+s+s1}{\PYZsq{}}\PY{p}{:} \PY{l+s+s1}{\PYZsq{}}\PY{l+s+s1}{./data/Washington\PYZhy{}2016\PYZhy{}Summary.csv}\PY{l+s+s1}{\PYZsq{}}\PY{p}{\PYZcb{}}\PY{p}{,}
                      \PY{l+s+s1}{\PYZsq{}}\PY{l+s+s1}{Chicago}\PY{l+s+s1}{\PYZsq{}}\PY{p}{:} \PY{p}{\PYZob{}}\PY{l+s+s1}{\PYZsq{}}\PY{l+s+s1}{in\PYZus{}file}\PY{l+s+s1}{\PYZsq{}}\PY{p}{:} \PY{l+s+s1}{\PYZsq{}}\PY{l+s+s1}{./data/Chicago\PYZhy{}Divvy\PYZhy{}2016.csv}\PY{l+s+s1}{\PYZsq{}}\PY{p}{,}
                                  \PY{l+s+s1}{\PYZsq{}}\PY{l+s+s1}{out\PYZus{}file}\PY{l+s+s1}{\PYZsq{}}\PY{p}{:} \PY{l+s+s1}{\PYZsq{}}\PY{l+s+s1}{./data/Chicago\PYZhy{}2016\PYZhy{}Summary.csv}\PY{l+s+s1}{\PYZsq{}}\PY{p}{\PYZcb{}}\PY{p}{,}
                      \PY{l+s+s1}{\PYZsq{}}\PY{l+s+s1}{NYC}\PY{l+s+s1}{\PYZsq{}}\PY{p}{:} \PY{p}{\PYZob{}}\PY{l+s+s1}{\PYZsq{}}\PY{l+s+s1}{in\PYZus{}file}\PY{l+s+s1}{\PYZsq{}}\PY{p}{:} \PY{l+s+s1}{\PYZsq{}}\PY{l+s+s1}{./data/NYC\PYZhy{}CitiBike\PYZhy{}2016.csv}\PY{l+s+s1}{\PYZsq{}}\PY{p}{,}
                              \PY{l+s+s1}{\PYZsq{}}\PY{l+s+s1}{out\PYZus{}file}\PY{l+s+s1}{\PYZsq{}}\PY{p}{:} \PY{l+s+s1}{\PYZsq{}}\PY{l+s+s1}{./data/NYC\PYZhy{}2016\PYZhy{}Summary.csv}\PY{l+s+s1}{\PYZsq{}}\PY{p}{\PYZcb{}}\PY{p}{\PYZcb{}}
         
         \PY{k}{for} \PY{n}{city}\PY{p}{,} \PY{n}{filenames} \PY{o+ow}{in} \PY{n}{city\PYZus{}info}\PY{o}{.}\PY{n}{items}\PY{p}{(}\PY{p}{)}\PY{p}{:}
             \PY{n}{condense\PYZus{}data}\PY{p}{(}\PY{n}{filenames}\PY{p}{[}\PY{l+s+s1}{\PYZsq{}}\PY{l+s+s1}{in\PYZus{}file}\PY{l+s+s1}{\PYZsq{}}\PY{p}{]}\PY{p}{,} \PY{n}{filenames}\PY{p}{[}\PY{l+s+s1}{\PYZsq{}}\PY{l+s+s1}{out\PYZus{}file}\PY{l+s+s1}{\PYZsq{}}\PY{p}{]}\PY{p}{,} \PY{n}{city}\PY{p}{)}
             \PY{n}{print\PYZus{}first\PYZus{}point}\PY{p}{(}\PY{n}{filenames}\PY{p}{[}\PY{l+s+s1}{\PYZsq{}}\PY{l+s+s1}{out\PYZus{}file}\PY{l+s+s1}{\PYZsq{}}\PY{p}{]}\PY{p}{)}
\end{Verbatim}


    \begin{Verbatim}[commandchars=\\\{\}]

City: Washington
OrderedDict([('duration', '7.123116666666666'),
             ('month', '3'),
             ('hour', '22'),
             ('day\_of\_week', 'Thursday'),
             ('user\_type', 'Subscriber')])

City: Chicago
OrderedDict([('duration', '15.433333333333334'),
             ('month', '3'),
             ('hour', '23'),
             ('day\_of\_week', 'Thursday'),
             ('user\_type', 'Subscriber')])

City: NYC
OrderedDict([('duration', '13.983333333333333'),
             ('month', '1'),
             ('hour', '0'),
             ('day\_of\_week', 'Friday'),
             ('user\_type', 'Customer')])

    \end{Verbatim}

    \begin{quote}
\textbf{Tip}: If you save a jupyter Notebook, the output from running
code blocks will also be saved. However, the state of your workspace
will be reset once a new session is started. Make sure that you run all
of the necessary code blocks from your previous session to reestablish
variables and functions before picking up where you last left off.
\end{quote}

 \#\# Exploratory Data Analysis

Now that you have the data collected and wrangled, you're ready to start
exploring the data. In this section you will write some code to compute
descriptive statistics from the data. You will also be introduced to the
\texttt{matplotlib} library to create some basic histograms of the data.

 \#\#\# Statistics

First, let's compute some basic counts. The first cell below contains a
function that uses the csv module to iterate through a provided data
file, returning the number of trips made by subscribers and customers.
The second cell runs this function on the example Bay Area data in the
\texttt{/examples/} folder. Modify the cells to answer the question
below.

\textbf{Question 4a}: Which city has the highest number of trips? Which
city has the highest proportion of trips made by subscribers? Which city
has the highest proportion of trips made by short-term customers?

\textbf{Answer}: According to analysis, \textbf{NYC city} having highest
subscribers, highest customers and highest trips

    \begin{Verbatim}[commandchars=\\\{\}]
{\color{incolor}In [{\color{incolor}28}]:} \PY{k}{def} \PY{n+nf}{number\PYZus{}of\PYZus{}trips}\PY{p}{(}\PY{n}{filename}\PY{p}{,}\PY{n}{city}\PY{p}{)}\PY{p}{:}
             \PY{l+s+sd}{\PYZdq{}\PYZdq{}\PYZdq{}}
         \PY{l+s+sd}{    This function reads in a file with trip data and reports the number of}
         \PY{l+s+sd}{    trips made by subscribers, customers, and total overall.}
         \PY{l+s+sd}{    \PYZdq{}\PYZdq{}\PYZdq{}}
             \PY{k}{with} \PY{n+nb}{open}\PY{p}{(}\PY{n}{filename}\PY{p}{,} \PY{l+s+s1}{\PYZsq{}}\PY{l+s+s1}{r}\PY{l+s+s1}{\PYZsq{}}\PY{p}{)} \PY{k}{as} \PY{n}{f\PYZus{}in}\PY{p}{:}
                 \PY{c+c1}{\PYZsh{} set up csv reader object}
                 \PY{n}{reader} \PY{o}{=} \PY{n}{csv}\PY{o}{.}\PY{n}{DictReader}\PY{p}{(}\PY{n}{f\PYZus{}in}\PY{p}{)}
                 
                 \PY{c+c1}{\PYZsh{} initialize count variables}
                 \PY{n}{n\PYZus{}subscribers} \PY{o}{=} \PY{l+m+mi}{0}
                 \PY{n}{n\PYZus{}customers} \PY{o}{=} \PY{l+m+mi}{0}
                 
                 \PY{c+c1}{\PYZsh{} tally up ride types}
                 \PY{k}{for} \PY{n}{row} \PY{o+ow}{in} \PY{n}{reader}\PY{p}{:}
                     \PY{k}{if} \PY{n}{row}\PY{p}{[}\PY{l+s+s1}{\PYZsq{}}\PY{l+s+s1}{user\PYZus{}type}\PY{l+s+s1}{\PYZsq{}}\PY{p}{]} \PY{o}{==} \PY{l+s+s1}{\PYZsq{}}\PY{l+s+s1}{Subscriber}\PY{l+s+s1}{\PYZsq{}}\PY{p}{:}
                         \PY{n}{n\PYZus{}subscribers} \PY{o}{+}\PY{o}{=} \PY{l+m+mi}{1}
                     \PY{k}{else}\PY{p}{:}
                         \PY{n}{n\PYZus{}customers} \PY{o}{+}\PY{o}{=} \PY{l+m+mi}{1}
                 
                 \PY{c+c1}{\PYZsh{} compute total number of rides}
                 \PY{n}{n\PYZus{}total} \PY{o}{=} \PY{n}{n\PYZus{}subscribers} \PY{o}{+} \PY{n}{n\PYZus{}customers}
                 
                 \PY{c+c1}{\PYZsh{} return tallies as a tuple}
                 \PY{k}{return}\PY{p}{(}\PY{n}{n\PYZus{}subscribers}\PY{p}{,} \PY{n}{n\PYZus{}customers}\PY{p}{,} \PY{n}{n\PYZus{}total}\PY{p}{)}
\end{Verbatim}


    \begin{Verbatim}[commandchars=\\\{\}]
{\color{incolor}In [{\color{incolor}29}]:} \PY{c+c1}{\PYZsh{}\PYZsh{} Modify this and the previous cell to answer Question 4a. Remember to run \PYZsh{}\PYZsh{}}
         \PY{c+c1}{\PYZsh{}\PYZsh{} the function on the cleaned data files you created from Question 3.      \PYZsh{}\PYZsh{}}
         
         \PY{c+c1}{\PYZsh{}data\PYZus{}file = \PYZsq{}./examples/BayArea\PYZhy{}Y3\PYZhy{}Summary.csv\PYZsq{}}
         \PY{c+c1}{\PYZsh{}print(number\PYZus{}of\PYZus{}trips(data\PYZus{}file)) \PYZsh{}(5666, 633, 6299)}
         
         \PY{n}{city\PYZus{}files} \PY{o}{=} \PY{p}{\PYZob{}}\PY{l+s+s1}{\PYZsq{}}\PY{l+s+s1}{Washington}\PY{l+s+s1}{\PYZsq{}}\PY{p}{:} \PY{l+s+s1}{\PYZsq{}}\PY{l+s+s1}{./data/Washington\PYZhy{}2016\PYZhy{}Summary.csv}\PY{l+s+s1}{\PYZsq{}}\PY{p}{,}
                      \PY{l+s+s1}{\PYZsq{}}\PY{l+s+s1}{Chicago}\PY{l+s+s1}{\PYZsq{}}\PY{p}{:} \PY{l+s+s1}{\PYZsq{}}\PY{l+s+s1}{./data/Chicago\PYZhy{}2016\PYZhy{}Summary.csv}\PY{l+s+s1}{\PYZsq{}}\PY{p}{,}
                      \PY{l+s+s1}{\PYZsq{}}\PY{l+s+s1}{NYC}\PY{l+s+s1}{\PYZsq{}}\PY{p}{:} \PY{l+s+s1}{\PYZsq{}}\PY{l+s+s1}{./data/NYC\PYZhy{}2016\PYZhy{}Summary.csv}\PY{l+s+s1}{\PYZsq{}}\PY{p}{\PYZcb{}}
         
         \PY{n}{n\PYZus{}subscribers}\PY{p}{,} \PY{n}{n\PYZus{}customers}\PY{p}{,} \PY{n}{n\PYZus{}trips} \PY{o}{=} \PY{l+m+mi}{0}\PY{p}{,} \PY{l+m+mi}{0}\PY{p}{,} \PY{l+m+mi}{0}
         \PY{n}{highest\PYZus{}subscribers}\PY{p}{,} \PY{n}{highest\PYZus{}customers}\PY{p}{,} \PY{n}{highest\PYZus{}trips} \PY{o}{=} \PY{l+m+mi}{0}\PY{p}{,} \PY{l+m+mi}{0}\PY{p}{,} \PY{l+m+mi}{0}
         \PY{n}{highest\PYZus{}subscribers\PYZus{}city}\PY{p}{,} \PY{n}{highest\PYZus{}customers\PYZus{}city}\PY{p}{,} \PY{n}{highest\PYZus{}trips\PYZus{}city} \PY{o}{=} \PY{l+m+mi}{0}\PY{p}{,} \PY{l+m+mi}{0}\PY{p}{,} \PY{l+m+mi}{0}
         
         \PY{k}{for} \PY{n}{city}\PY{p}{,}\PY{n}{filenames} \PY{o+ow}{in} \PY{n}{city\PYZus{}files}\PY{o}{.}\PY{n}{items}\PY{p}{(}\PY{p}{)}\PY{p}{:}
             \PY{n}{n\PYZus{}subscribers}\PY{p}{,} \PY{n}{n\PYZus{}customers}\PY{p}{,} \PY{n}{n\PYZus{}trips} \PY{o}{=} \PY{n}{number\PYZus{}of\PYZus{}trips}\PY{p}{(}\PY{n}{filenames}\PY{p}{,}\PY{n}{city}\PY{p}{)}
             \PY{n+nb}{print}\PY{p}{(}\PY{l+s+s2}{\PYZdq{}}\PY{l+s+s2}{City : }\PY{l+s+s2}{\PYZdq{}}\PY{o}{+}\PY{n+nb}{str}\PY{p}{(}\PY{n}{city}\PY{p}{)}\PY{o}{+}\PY{l+s+s2}{\PYZdq{}}\PY{l+s+s2}{ || n\PYZus{}subscribers: }\PY{l+s+s2}{\PYZdq{}}\PY{o}{+}\PY{n+nb}{str}\PY{p}{(}\PY{n}{n\PYZus{}subscribers}\PY{p}{)}\PY{o}{+}\PY{l+s+s2}{\PYZdq{}}\PY{l+s+s2}{ n\PYZus{}customers: }\PY{l+s+s2}{\PYZdq{}}\PY{o}{+}\PY{n+nb}{str}\PY{p}{(}\PY{n}{n\PYZus{}customers}\PY{p}{)}\PY{o}{+} \PY{l+s+s2}{\PYZdq{}}\PY{l+s+s2}{ n\PYZus{}trips: }\PY{l+s+s2}{\PYZdq{}}\PY{o}{+}\PY{n+nb}{str}\PY{p}{(}\PY{n}{n\PYZus{}trips}\PY{p}{)}\PY{p}{)}
            
             \PY{k}{if} \PY{p}{(}\PY{n}{highest\PYZus{}subscribers} \PY{o}{\PYZlt{}} \PY{n}{n\PYZus{}subscribers}\PY{p}{)}\PY{p}{:}
                 \PY{n}{highest\PYZus{}subscribers} \PY{o}{=} \PY{n}{n\PYZus{}subscribers}
                 \PY{n}{highest\PYZus{}subscribers\PYZus{}city} \PY{o}{=} \PY{n}{city}
                 
             \PY{k}{if} \PY{p}{(}\PY{n}{highest\PYZus{}customers} \PY{o}{\PYZlt{}} \PY{n}{n\PYZus{}customers}\PY{p}{)}\PY{p}{:}
                 \PY{n}{highest\PYZus{}customers} \PY{o}{=} \PY{n}{n\PYZus{}customers}
                 \PY{n}{highest\PYZus{}customers\PYZus{}city} \PY{o}{=} \PY{n}{city}
                 
             \PY{k}{if} \PY{p}{(}\PY{n}{highest\PYZus{}trips} \PY{o}{\PYZlt{}} \PY{n}{n\PYZus{}trips}\PY{p}{)}\PY{p}{:}
                 \PY{n}{highest\PYZus{}trips} \PY{o}{=} \PY{n}{n\PYZus{}trips}
                 \PY{n}{highest\PYZus{}trips\PYZus{}city} \PY{o}{=} \PY{n}{city}
         
         
         \PY{c+c1}{\PYZsh{}print(\PYZdq{}highest\PYZus{}subscribers: \PYZdq{}+str(highest\PYZus{}subscribers)+ \PYZdq{} City: \PYZdq{}+ str(highest\PYZus{}subscribers\PYZus{}city))}
         \PY{c+c1}{\PYZsh{}print(\PYZdq{}highest\PYZus{}customers: \PYZdq{}+str(highest\PYZus{}customers)+\PYZdq{} City: \PYZdq{}+ str(highest\PYZus{}customers\PYZus{}city))}
         \PY{c+c1}{\PYZsh{}print(\PYZdq{}highest\PYZus{}trips: \PYZdq{}+str(highest\PYZus{}trips)+ \PYZdq{} City: \PYZdq{}+ str(highest\PYZus{}trips\PYZus{}city))}
\end{Verbatim}


    \begin{Verbatim}[commandchars=\\\{\}]
City : Washington || n\_subscribers: 51753 n\_customers: 14573 n\_trips: 66326
City : Chicago || n\_subscribers: 54982 n\_customers: 17149 n\_trips: 72131
City : NYC || n\_subscribers: 245896 n\_customers: 30902 n\_trips: 276798
highest\_subscribers: 245896 City: NYC
highest\_customers: 30902 City: NYC
highest\_trips: 276798 City: NYC

    \end{Verbatim}

    \begin{quote}
\textbf{Tip}: In order to add additional cells to a notebook, you can
use the ``Insert Cell Above'' and ``Insert Cell Below'' options from the
menu bar above. There is also an icon in the toolbar for adding new
cells, with additional icons for moving the cells up and down the
document. By default, new cells are of the code type; you can also
specify the cell type (e.g.~Code or Markdown) of selected cells from the
Cell menu or the dropdown in the toolbar.
\end{quote}

Now, you will write your own code to continue investigating properties
of the data.

\textbf{Question 4b}: Bike-share systems are designed for riders to take
short trips. Most of the time, users are allowed to take trips of 30
minutes or less with no additional charges, with overage charges made
for trips of longer than that duration. What is the average trip length
for each city? What proportion of rides made in each city are longer
than 30 minutes?

\textbf{Answer}: \textbf{Washington} having avg\_trip\_length:
\textbf{14} pro\_of\_overchages: \textbf{8.0} \textbar{}\textbar{}
\textbf{Chicago} having avg\_trip\_length: \textbf{14}
pro\_of\_overchages: \textbf{6.9} \textbar{}\textbar{} \textbf{NYC}
having avg\_trip\_length: \textbf{13} pro\_of\_overchages: \textbf{6.6}

    \begin{Verbatim}[commandchars=\\\{\}]
{\color{incolor}In [{\color{incolor}30}]:} \PY{k}{def} \PY{n+nf}{avg\PYZus{}length\PYZus{}pro\PYZus{}of\PYZus{}overchages}\PY{p}{(}\PY{n}{filename}\PY{p}{)}\PY{p}{:}
             \PY{l+s+sd}{\PYZdq{}\PYZdq{}\PYZdq{}}
         \PY{l+s+sd}{    This function reads in a file with trip data and reports the number of}
         \PY{l+s+sd}{    trips made by subscribers, customers, and total overall.}
         \PY{l+s+sd}{    \PYZdq{}\PYZdq{}\PYZdq{}}
             \PY{k}{with} \PY{n+nb}{open}\PY{p}{(}\PY{n}{filename}\PY{p}{,} \PY{l+s+s1}{\PYZsq{}}\PY{l+s+s1}{r}\PY{l+s+s1}{\PYZsq{}}\PY{p}{)} \PY{k}{as} \PY{n}{f\PYZus{}in}\PY{p}{:}
                 \PY{c+c1}{\PYZsh{} set up csv reader object}
                 \PY{n}{reader} \PY{o}{=} \PY{n}{csv}\PY{o}{.}\PY{n}{DictReader}\PY{p}{(}\PY{n}{f\PYZus{}in}\PY{p}{)}
                 
                 \PY{c+c1}{\PYZsh{} initialize count variables}
                 \PY{n}{total\PYZus{}trip\PYZus{}length} \PY{o}{=} \PY{l+m+mi}{0}
                 \PY{n}{pro\PYZus{}of\PYZus{}overchages} \PY{o}{=} \PY{l+m+mi}{0}
                 \PY{n}{n\PYZus{}trips} \PY{o}{=} \PY{l+m+mi}{0}
                 
                 \PY{c+c1}{\PYZsh{} tally up duration details}
                 \PY{k}{for} \PY{n}{row} \PY{o+ow}{in} \PY{n}{reader}\PY{p}{:}
                     \PY{n}{n\PYZus{}trips} \PY{o}{+}\PY{o}{=} \PY{l+m+mi}{1}
                     \PY{n}{total\PYZus{}trip\PYZus{}length}  \PY{o}{+}\PY{o}{=} \PY{n+nb}{float}\PY{p}{(}\PY{n}{row}\PY{p}{[}\PY{l+s+s1}{\PYZsq{}}\PY{l+s+s1}{duration}\PY{l+s+s1}{\PYZsq{}}\PY{p}{]}\PY{p}{)}
                     \PY{k}{if} \PY{p}{(}\PY{n+nb}{float}\PY{p}{(}\PY{n}{row}\PY{p}{[}\PY{l+s+s1}{\PYZsq{}}\PY{l+s+s1}{duration}\PY{l+s+s1}{\PYZsq{}}\PY{p}{]}\PY{p}{)} \PY{o}{\PYZgt{}} \PY{l+m+mi}{30}\PY{p}{)}\PY{p}{:}
                         \PY{n}{pro\PYZus{}of\PYZus{}overchages} \PY{o}{+}\PY{o}{=}\PY{l+m+mi}{1}
                 
                 \PY{c+c1}{\PYZsh{} average trip length}
                 \PY{n}{avg\PYZus{}trip\PYZus{}length} \PY{o}{=} \PY{n}{total\PYZus{}trip\PYZus{}length} \PY{o}{/} \PY{n}{n\PYZus{}trips}
                 
                 \PY{c+c1}{\PYZsh{} proportion of rides made in each city are longer than 30 minutes}
                 \PY{n}{pro\PYZus{}of\PYZus{}overchages} \PY{o}{=} \PY{p}{(}\PY{n}{pro\PYZus{}of\PYZus{}overchages} \PY{o}{*} \PY{l+m+mi}{100}\PY{p}{)} \PY{o}{/} \PY{n}{n\PYZus{}trips}
                 
                 \PY{c+c1}{\PYZsh{} return total\PYZus{}trip\PYZus{}length and pro\PYZus{}of\PYZus{}overchages as a tuple}
                 \PY{k}{return}\PY{p}{(}\PY{n}{avg\PYZus{}trip\PYZus{}length}\PY{p}{,} \PY{n}{pro\PYZus{}of\PYZus{}overchages}\PY{p}{)}
\end{Verbatim}


    \begin{Verbatim}[commandchars=\\\{\}]
{\color{incolor}In [{\color{incolor}34}]:} \PY{c+c1}{\PYZsh{}\PYZsh{} Use this and additional cells to answer Question 4b.                 \PYZsh{}\PYZsh{}}
         \PY{c+c1}{\PYZsh{}\PYZsh{}                                                                      \PYZsh{}\PYZsh{}}
         \PY{c+c1}{\PYZsh{}\PYZsh{} HINT: The csv module reads in all of the data as strings, including  \PYZsh{}\PYZsh{}}
         \PY{c+c1}{\PYZsh{}\PYZsh{} numeric values. You will need a function to convert the strings      \PYZsh{}\PYZsh{}}
         \PY{c+c1}{\PYZsh{}\PYZsh{} into an appropriate numeric type before you aggregate data.          \PYZsh{}\PYZsh{}}
         \PY{c+c1}{\PYZsh{}\PYZsh{} TIP: For the Bay Area example, the average trip length is 14 minutes \PYZsh{}\PYZsh{}}
         \PY{c+c1}{\PYZsh{}\PYZsh{} and 3.5\PYZpc{} of trips are longer than 30 minutes.}
         \PY{c+c1}{\PYZsh{}\PYZsh{}}
         \PY{c+c1}{\PYZsh{}data\PYZus{}file = \PYZsq{}./examples/BayArea\PYZhy{}Y3\PYZhy{}Summary.csv\PYZsq{}}
         \PY{c+c1}{\PYZsh{}avg\PYZus{}trip\PYZus{}length, pro\PYZus{}of\PYZus{}overchages = avg\PYZus{}length\PYZus{}pro\PYZus{}of\PYZus{}overchages(data\PYZus{}file)}
         \PY{c+c1}{\PYZsh{}print(\PYZsq{}total\PYZus{}trip\PYZus{}length: \PYZsq{}+str(int(avg\PYZus{}trip\PYZus{}length))+ \PYZsq{} pro\PYZus{}of\PYZus{}overchages: \PYZsq{}+str(format(pro\PYZus{}of\PYZus{}overchages, \PYZsq{}.1f\PYZsq{})) )}
         
         \PY{n}{city\PYZus{}files} \PY{o}{=} \PY{p}{\PYZob{}}\PY{l+s+s1}{\PYZsq{}}\PY{l+s+s1}{Washington}\PY{l+s+s1}{\PYZsq{}}\PY{p}{:} \PY{l+s+s1}{\PYZsq{}}\PY{l+s+s1}{./data/Washington\PYZhy{}2016\PYZhy{}Summary.csv}\PY{l+s+s1}{\PYZsq{}}\PY{p}{,}
                      \PY{l+s+s1}{\PYZsq{}}\PY{l+s+s1}{Chicago}\PY{l+s+s1}{\PYZsq{}}\PY{p}{:} \PY{l+s+s1}{\PYZsq{}}\PY{l+s+s1}{./data/Chicago\PYZhy{}2016\PYZhy{}Summary.csv}\PY{l+s+s1}{\PYZsq{}}\PY{p}{,}
                      \PY{l+s+s1}{\PYZsq{}}\PY{l+s+s1}{NYC}\PY{l+s+s1}{\PYZsq{}}\PY{p}{:} \PY{l+s+s1}{\PYZsq{}}\PY{l+s+s1}{./data/NYC\PYZhy{}2016\PYZhy{}Summary.csv}\PY{l+s+s1}{\PYZsq{}}\PY{p}{\PYZcb{}}
         
         \PY{n}{total\PYZus{}trip\PYZus{}length}\PY{p}{,} \PY{n}{pro\PYZus{}of\PYZus{}overchages} \PY{o}{=} \PY{l+m+mi}{0}\PY{p}{,} \PY{l+m+mi}{0}
         
         \PY{k}{for} \PY{n}{city}\PY{p}{,}\PY{n}{filenames} \PY{o+ow}{in} \PY{n}{city\PYZus{}files}\PY{o}{.}\PY{n}{items}\PY{p}{(}\PY{p}{)}\PY{p}{:}
             \PY{n}{avg\PYZus{}trip\PYZus{}length}\PY{p}{,} \PY{n}{pro\PYZus{}of\PYZus{}overchages} \PY{o}{=} \PY{n}{avg\PYZus{}length\PYZus{}pro\PYZus{}of\PYZus{}overchages}\PY{p}{(}\PY{n}{filenames}\PY{p}{)}
             \PY{n}{avg\PYZus{}trip\PYZus{}length}\PY{p}{,} \PY{n}{pro\PYZus{}of\PYZus{}overchages} \PY{o}{=} \PY{n+nb}{int}\PY{p}{(}\PY{n}{avg\PYZus{}trip\PYZus{}length}\PY{p}{)}\PY{p}{,} \PY{n+nb}{format}\PY{p}{(}\PY{n}{pro\PYZus{}of\PYZus{}overchages}\PY{p}{,} \PY{l+s+s1}{\PYZsq{}}\PY{l+s+s1}{.1f}\PY{l+s+s1}{\PYZsq{}}\PY{p}{)}
             \PY{c+c1}{\PYZsh{}print(str(city)+\PYZsq{} having avg\PYZus{}trip\PYZus{}length: \PYZsq{}+str(avg\PYZus{}trip\PYZus{}length)+ \PYZsq{} pro\PYZus{}of\PYZus{}overchages: \PYZsq{}+pro\PYZus{}of\PYZus{}overchages)}
\end{Verbatim}


    \textbf{Question 4c}: Dig deeper into the question of trip duration
based on ridership. Choose one city. Within that city, which type of
user takes longer rides on average: Subscribers or Customers?

\textbf{Answer}: In \textbf{NYC city} \textbf{Customers} takes longer
rides on average

    \begin{Verbatim}[commandchars=\\\{\}]
{\color{incolor}In [{\color{incolor}2}]:} \PY{k}{def} \PY{n+nf}{find\PYZus{}longer\PYZus{}rides\PYZus{}type\PYZus{}user}\PY{p}{(}\PY{n}{filename}\PY{p}{)}\PY{p}{:}
            \PY{c+c1}{\PYZsh{}print(city)}
            \PY{k}{with} \PY{n+nb}{open}\PY{p}{(}\PY{n}{filename}\PY{p}{,}\PY{l+s+s1}{\PYZsq{}}\PY{l+s+s1}{r}\PY{l+s+s1}{\PYZsq{}}\PY{p}{)} \PY{k}{as} \PY{n}{f\PYZus{}in}\PY{p}{:}
                \PY{n}{reader} \PY{o}{=} \PY{n}{csv}\PY{o}{.}\PY{n}{DictReader}\PY{p}{(}\PY{n}{f\PYZus{}in}\PY{p}{)}
                \PY{n}{Subscribers} \PY{o}{=} \PY{l+m+mi}{0}
                \PY{n}{Customers} \PY{o}{=} \PY{l+m+mi}{0}
                \PY{n}{total\PYZus{}duration\PYZus{}s} \PY{o}{=} \PY{l+m+mi}{0}
                \PY{n}{total\PYZus{}duration\PYZus{}c} \PY{o}{=} \PY{l+m+mi}{0}
                \PY{k}{for} \PY{n}{row} \PY{o+ow}{in} \PY{n}{reader}\PY{p}{:}
                    \PY{k}{if} \PY{p}{(}\PY{n}{row}\PY{p}{[}\PY{l+s+s1}{\PYZsq{}}\PY{l+s+s1}{user\PYZus{}type}\PY{l+s+s1}{\PYZsq{}}\PY{p}{]} \PY{o}{==} \PY{l+s+s1}{\PYZsq{}}\PY{l+s+s1}{Subscriber}\PY{l+s+s1}{\PYZsq{}}\PY{p}{)}\PY{p}{:}
                        \PY{n}{Subscribers} \PY{o}{+}\PY{o}{=} \PY{l+m+mi}{1}
                        \PY{n}{total\PYZus{}duration\PYZus{}s} \PY{o}{+}\PY{o}{=} \PY{n+nb}{float}\PY{p}{(}\PY{n}{row}\PY{p}{[}\PY{l+s+s1}{\PYZsq{}}\PY{l+s+s1}{duration}\PY{l+s+s1}{\PYZsq{}}\PY{p}{]}\PY{p}{)}
                    \PY{k}{if} \PY{p}{(}\PY{n}{row}\PY{p}{[}\PY{l+s+s1}{\PYZsq{}}\PY{l+s+s1}{user\PYZus{}type}\PY{l+s+s1}{\PYZsq{}}\PY{p}{]} \PY{o}{==} \PY{l+s+s1}{\PYZsq{}}\PY{l+s+s1}{Customer}\PY{l+s+s1}{\PYZsq{}}\PY{p}{)}\PY{p}{:}
                        \PY{n}{Customers} \PY{o}{+}\PY{o}{=} \PY{l+m+mi}{1}
                        \PY{n}{total\PYZus{}duration\PYZus{}c} \PY{o}{+}\PY{o}{=} \PY{n+nb}{float}\PY{p}{(}\PY{n}{row}\PY{p}{[}\PY{l+s+s1}{\PYZsq{}}\PY{l+s+s1}{duration}\PY{l+s+s1}{\PYZsq{}}\PY{p}{]}\PY{p}{)}
                
                \PY{n}{avg\PYZus{}long\PYZus{}rides\PYZus{}Subscriber} \PY{o}{=}  \PY{n}{total\PYZus{}duration\PYZus{}s} \PY{o}{/} \PY{n}{Subscribers}
                \PY{n}{avg\PYZus{}long\PYZus{}rides\PYZus{}Customers} \PY{o}{=}  \PY{n}{total\PYZus{}duration\PYZus{}c} \PY{o}{/} \PY{n}{Customers}
                \PY{n+nb}{print}\PY{p}{(}\PY{n}{avg\PYZus{}long\PYZus{}rides\PYZus{}Subscriber}\PY{p}{)}
                \PY{n+nb}{print}\PY{p}{(}\PY{n}{avg\PYZus{}long\PYZus{}rides\PYZus{}Customers}\PY{p}{)}
            \PY{c+c1}{\PYZsh{}return avg\PYZus{}long\PYZus{}rides\PYZus{}Subscriber, avg\PYZus{}long\PYZus{}rides\PYZus{}Customers}
            \PY{k}{return} \PY{n+nb}{format}\PY{p}{(}\PY{n}{avg\PYZus{}long\PYZus{}rides\PYZus{}Subscriber}\PY{p}{,} \PY{l+s+s1}{\PYZsq{}}\PY{l+s+s1}{.1f}\PY{l+s+s1}{\PYZsq{}}\PY{p}{)}\PY{p}{,} \PY{n+nb}{format}\PY{p}{(}\PY{n}{avg\PYZus{}long\PYZus{}rides\PYZus{}Customers}\PY{p}{,} \PY{l+s+s1}{\PYZsq{}}\PY{l+s+s1}{.1f}\PY{l+s+s1}{\PYZsq{}}\PY{p}{)}
\end{Verbatim}


    \begin{Verbatim}[commandchars=\\\{\}]
{\color{incolor}In [{\color{incolor}3}]:} \PY{c+c1}{\PYZsh{}\PYZsh{} Use this and additional cells to answer Question 4c. If you have    \PYZsh{}\PYZsh{}}
        \PY{c+c1}{\PYZsh{}\PYZsh{} not done so yet, consider revising some of your previous code to    \PYZsh{}\PYZsh{}}
        \PY{c+c1}{\PYZsh{}\PYZsh{} make use of functions for reusability.                              \PYZsh{}\PYZsh{}}
        \PY{c+c1}{\PYZsh{}\PYZsh{}                                                                     \PYZsh{}\PYZsh{}}
        \PY{c+c1}{\PYZsh{}\PYZsh{} TIP: For the Bay Area example data, you should find the average     \PYZsh{}\PYZsh{}}
        \PY{c+c1}{\PYZsh{}\PYZsh{} Subscriber trip duration to be 9.5 minutes and the average Customer \PYZsh{}\PYZsh{}}
        \PY{c+c1}{\PYZsh{}\PYZsh{} trip duration to be 54.6 minutes. Do the other cities have this     \PYZsh{}\PYZsh{}}
        \PY{c+c1}{\PYZsh{}\PYZsh{} level of difference?                                                \PYZsh{}\PYZsh{}}
        
        \PY{c+c1}{\PYZsh{}data\PYZus{}file = \PYZsq{}./examples/BayArea\PYZhy{}Y3\PYZhy{}Summary.csv\PYZsq{}}
        \PY{c+c1}{\PYZsh{}print(find\PYZus{}longer\PYZus{}rides\PYZus{}type\PYZus{}user(data\PYZus{}file))}
        
        
        \PY{c+c1}{\PYZsh{} as mentioned in let\PYZsq{}s choose any one city i.e. NY}
        \PY{n}{data\PYZus{}file\PYZus{}NYC} \PY{o}{=} \PY{p}{\PYZob{}}\PY{l+s+s1}{\PYZsq{}}\PY{l+s+s1}{NYC}\PY{l+s+s1}{\PYZsq{}}\PY{p}{:} \PY{l+s+s1}{\PYZsq{}}\PY{l+s+s1}{./data/NYC\PYZhy{}2016\PYZhy{}Summary.csv}\PY{l+s+s1}{\PYZsq{}}\PY{p}{\PYZcb{}}
        
        \PY{n}{avg\PYZus{}long\PYZus{}rides\PYZus{}Subscriber}\PY{p}{,} \PY{n}{avg\PYZus{}long\PYZus{}rides\PYZus{}Customers} \PY{o}{=}  \PY{l+m+mf}{0.0}\PY{p}{,} \PY{l+m+mf}{0.0}
        
        \PY{n}{avg\PYZus{}long\PYZus{}rides\PYZus{}Subscriber}\PY{p}{,} \PY{n}{avg\PYZus{}long\PYZus{}rides\PYZus{}Customers} \PY{o}{=} \PY{n}{find\PYZus{}longer\PYZus{}rides\PYZus{}type\PYZus{}user}\PY{p}{(}\PY{n}{data\PYZus{}file\PYZus{}NYC}\PY{p}{[}\PY{l+s+s1}{\PYZsq{}}\PY{l+s+s1}{NYC}\PY{l+s+s1}{\PYZsq{}}\PY{p}{]}\PY{p}{)}
        
        \PY{n+nb}{print}\PY{p}{(}\PY{l+s+s1}{\PYZsq{}}\PY{l+s+s1}{NYC:: avg\PYZus{}long\PYZus{}rides\PYZus{}Subscriber: }\PY{l+s+s1}{\PYZsq{}}\PY{o}{+}\PY{n+nb}{str}\PY{p}{(}\PY{n}{avg\PYZus{}long\PYZus{}rides\PYZus{}Subscriber}\PY{p}{)}\PY{o}{+} \PY{l+s+s1}{\PYZsq{}}\PY{l+s+s1}{ avg\PYZus{}long\PYZus{}rides\PYZus{}Customers: }\PY{l+s+s1}{\PYZsq{}}\PY{o}{+}\PY{n+nb}{str}\PY{p}{(}\PY{n}{avg\PYZus{}long\PYZus{}rides\PYZus{}Customers}\PY{p}{)}\PY{p}{)}
        
        \PY{k}{if} \PY{p}{(}\PY{n}{avg\PYZus{}long\PYZus{}rides\PYZus{}Subscriber} \PY{o}{\PYZgt{}} \PY{n}{avg\PYZus{}long\PYZus{}rides\PYZus{}Customers}\PY{p}{)}\PY{p}{:}
            \PY{n+nb}{print}\PY{p}{(}\PY{l+s+s2}{\PYZdq{}}\PY{l+s+s2}{Subscriber takes longer rides on average}\PY{l+s+s2}{\PYZdq{}}\PY{p}{)}
        \PY{k}{else}\PY{p}{:}
            \PY{n+nb}{print}\PY{p}{(}\PY{l+s+s2}{\PYZdq{}}\PY{l+s+s2}{Customers takes longer rides on average}\PY{l+s+s2}{\PYZdq{}}\PY{p}{)}   
        
        \PY{c+c1}{\PYZsh{}TESTED FOR ALL OTHER CITIES}
        
        \PY{c+c1}{\PYZsh{}city\PYZus{}files = \PYZob{}\PYZsq{}Washington\PYZsq{}: \PYZsq{}./data/Washington\PYZhy{}2016\PYZhy{}Summary.csv\PYZsq{},}
                     \PY{c+c1}{\PYZsh{}\PYZsq{}Chicago\PYZsq{}: \PYZsq{}./data/Chicago\PYZhy{}2016\PYZhy{}Summary.csv\PYZsq{},}
                     \PY{c+c1}{\PYZsh{}\PYZsq{}NYC\PYZsq{}: \PYZsq{}./data/NYC\PYZhy{}2016\PYZhy{}Summary.csv\PYZsq{}\PYZcb{}}
        
        \PY{c+c1}{\PYZsh{}avg\PYZus{}long\PYZus{}rides\PYZus{}Subscriber, avg\PYZus{}long\PYZus{}rides\PYZus{}Customers =  0, 0 }
        
        \PY{c+c1}{\PYZsh{}for city,filenames in city\PYZus{}files.items():}
            \PY{c+c1}{\PYZsh{}avg\PYZus{}long\PYZus{}rides\PYZus{}Subscriber, avg\PYZus{}long\PYZus{}rides\PYZus{}Customers = find\PYZus{}longer\PYZus{}rides\PYZus{}type\PYZus{}user(filenames)}
            \PY{c+c1}{\PYZsh{}print(str(city) +\PYZsq{}:: avg\PYZus{}long\PYZus{}rides\PYZus{}Subscriber: \PYZsq{}+str(avg\PYZus{}long\PYZus{}rides\PYZus{}Subscriber)+ \PYZsq{} avg\PYZus{}long\PYZus{}rides\PYZus{}Customers: \PYZsq{}+str(avg\PYZus{}long\PYZus{}rides\PYZus{}Customers))}
            \PY{c+c1}{\PYZsh{}if (avg\PYZus{}long\PYZus{}rides\PYZus{}Subscriber \PYZgt{} avg\PYZus{}long\PYZus{}rides\PYZus{}Customers):}
                \PY{c+c1}{\PYZsh{}print(\PYZdq{}Subscriber takes longer rides on average\PYZdq{})}
            \PY{c+c1}{\PYZsh{}else:}
                \PY{c+c1}{\PYZsh{}print(\PYZdq{}Customers takes longer rides on average\PYZdq{})}
\end{Verbatim}


    \begin{Verbatim}[commandchars=\\\{\}]
12.206264708115235
21.49881729335759
NYC:: avg\_long\_rides\_Subscriber: 12.2 avg\_long\_rides\_Customers: 21.5
Customers takes longer rides on average

    \end{Verbatim}

     \#\#\# Visualizations

The last set of values that you computed should have pulled up an
interesting result. While the mean trip time for Subscribers is well
under 30 minutes, the mean trip time for Customers is actually
\emph{above} 30 minutes! It will be interesting for us to look at how
the trip times are distributed. In order to do this, a new library will
be introduced here, \texttt{matplotlib}. Run the cell below to load the
library and to generate an example plot.

    \begin{Verbatim}[commandchars=\\\{\}]
{\color{incolor}In [{\color{incolor}4}]:} \PY{c+c1}{\PYZsh{} load library}
        \PY{k+kn}{import} \PY{n+nn}{matplotlib}\PY{n+nn}{.}\PY{n+nn}{pyplot} \PY{k}{as} \PY{n+nn}{plt}
        
        \PY{c+c1}{\PYZsh{} this is a \PYZsq{}magic word\PYZsq{} that allows for plots to be displayed}
        \PY{c+c1}{\PYZsh{} inline with the notebook. If you want to know more, see:}
        \PY{c+c1}{\PYZsh{} http://ipython.readthedocs.io/en/stable/interactive/magics.html}
        \PY{o}{\PYZpc{}}\PY{k}{matplotlib} inline 
        
        \PY{c+c1}{\PYZsh{} example histogram, data taken from bay area sample}
        \PY{n}{data} \PY{o}{=} \PY{p}{[} \PY{l+m+mf}{7.65}\PY{p}{,}  \PY{l+m+mf}{8.92}\PY{p}{,}  \PY{l+m+mf}{7.42}\PY{p}{,}  \PY{l+m+mf}{5.50}\PY{p}{,} \PY{l+m+mf}{16.17}\PY{p}{,}  \PY{l+m+mf}{4.20}\PY{p}{,}  \PY{l+m+mf}{8.98}\PY{p}{,}  \PY{l+m+mf}{9.62}\PY{p}{,} \PY{l+m+mf}{11.48}\PY{p}{,} \PY{l+m+mf}{14.33}\PY{p}{,}
                \PY{l+m+mf}{19.02}\PY{p}{,} \PY{l+m+mf}{21.53}\PY{p}{,}  \PY{l+m+mf}{3.90}\PY{p}{,}  \PY{l+m+mf}{7.97}\PY{p}{,}  \PY{l+m+mf}{2.62}\PY{p}{,}  \PY{l+m+mf}{2.67}\PY{p}{,}  \PY{l+m+mf}{3.08}\PY{p}{,} \PY{l+m+mf}{14.40}\PY{p}{,} \PY{l+m+mf}{12.90}\PY{p}{,}  \PY{l+m+mf}{7.83}\PY{p}{,}
                \PY{l+m+mf}{25.12}\PY{p}{,}  \PY{l+m+mf}{8.30}\PY{p}{,}  \PY{l+m+mf}{4.93}\PY{p}{,} \PY{l+m+mf}{12.43}\PY{p}{,} \PY{l+m+mf}{10.60}\PY{p}{,}  \PY{l+m+mf}{6.17}\PY{p}{,} \PY{l+m+mf}{10.88}\PY{p}{,}  \PY{l+m+mf}{4.78}\PY{p}{,} \PY{l+m+mf}{15.15}\PY{p}{,}  \PY{l+m+mf}{3.53}\PY{p}{,}
                 \PY{l+m+mf}{9.43}\PY{p}{,} \PY{l+m+mf}{13.32}\PY{p}{,} \PY{l+m+mf}{11.72}\PY{p}{,}  \PY{l+m+mf}{9.85}\PY{p}{,}  \PY{l+m+mf}{5.22}\PY{p}{,} \PY{l+m+mf}{15.10}\PY{p}{,}  \PY{l+m+mf}{3.95}\PY{p}{,}  \PY{l+m+mf}{3.17}\PY{p}{,}  \PY{l+m+mf}{8.78}\PY{p}{,}  \PY{l+m+mf}{1.88}\PY{p}{,}
                 \PY{l+m+mf}{4.55}\PY{p}{,} \PY{l+m+mf}{12.68}\PY{p}{,} \PY{l+m+mf}{12.38}\PY{p}{,}  \PY{l+m+mf}{9.78}\PY{p}{,}  \PY{l+m+mf}{7.63}\PY{p}{,}  \PY{l+m+mf}{6.45}\PY{p}{,} \PY{l+m+mf}{17.38}\PY{p}{,} \PY{l+m+mf}{11.90}\PY{p}{,} \PY{l+m+mf}{11.52}\PY{p}{,}  \PY{l+m+mf}{8.63}\PY{p}{,}\PY{p}{]}
        \PY{n}{plt}\PY{o}{.}\PY{n}{hist}\PY{p}{(}\PY{n}{data}\PY{p}{)}
        \PY{n}{plt}\PY{o}{.}\PY{n}{title}\PY{p}{(}\PY{l+s+s1}{\PYZsq{}}\PY{l+s+s1}{Distribution of Trip Durations}\PY{l+s+s1}{\PYZsq{}}\PY{p}{)}
        \PY{n}{plt}\PY{o}{.}\PY{n}{xlabel}\PY{p}{(}\PY{l+s+s1}{\PYZsq{}}\PY{l+s+s1}{Duration (m)}\PY{l+s+s1}{\PYZsq{}}\PY{p}{)}
        \PY{n}{plt}\PY{o}{.}\PY{n}{show}\PY{p}{(}\PY{p}{)}
\end{Verbatim}


    \begin{center}
    \adjustimage{max size={0.9\linewidth}{0.9\paperheight}}{output_21_0.png}
    \end{center}
    { \hspace*{\fill} \\}
    
    In the above cell, we collected fifty trip times in a list, and passed
this list as the first argument to the \texttt{.hist()} function. This
function performs the computations and creates plotting objects for
generating a histogram, but the plot is actually not rendered until the
\texttt{.show()} function is executed. The \texttt{.title()} and
\texttt{.xlabel()} functions provide some labeling for plot context.

You will now use these functions to create a histogram of the trip times
for the city you selected in question 4c. Don't separate the Subscribers
and Customers for now: just collect all of the trip times and plot them.

    \begin{Verbatim}[commandchars=\\\{\}]
{\color{incolor}In [{\color{incolor}5}]:} \PY{k}{def} \PY{n+nf}{collect\PYZus{}trip\PYZus{}time}\PY{p}{(}\PY{n}{filename}\PY{p}{)}\PY{p}{:}
            \PY{l+s+sd}{\PYZdq{}\PYZdq{}\PYZdq{}}
        \PY{l+s+sd}{    functions to create a list of the trip times for the city that is selected in question 4c. (i.e NYC)}
        \PY{l+s+sd}{    \PYZdq{}\PYZdq{}\PYZdq{}}
            
            \PY{k}{with} \PY{n+nb}{open}\PY{p}{(}\PY{n}{filename}\PY{p}{,}\PY{l+s+s1}{\PYZsq{}}\PY{l+s+s1}{r}\PY{l+s+s1}{\PYZsq{}}\PY{p}{)} \PY{k}{as} \PY{n}{f\PYZus{}in}\PY{p}{:}
                
                \PY{n}{reader} \PY{o}{=} \PY{n}{csv}\PY{o}{.}\PY{n}{DictReader}\PY{p}{(}\PY{n}{f\PYZus{}in}\PY{p}{)}
                
                \PY{n}{trip\PYZus{}time} \PY{o}{=} \PY{p}{[}\PY{p}{]}
                \PY{k}{for} \PY{n}{row} \PY{o+ow}{in} \PY{n}{reader}\PY{p}{:}
                    \PY{n}{get\PYZus{}value} \PY{o}{=} \PY{n+nb}{float}\PY{p}{(}\PY{n}{row}\PY{p}{[}\PY{l+s+s1}{\PYZsq{}}\PY{l+s+s1}{duration}\PY{l+s+s1}{\PYZsq{}}\PY{p}{]}\PY{p}{)}
                    \PY{n}{getvalue} \PY{o}{=} \PY{n+nb}{format}\PY{p}{(}\PY{n}{get\PYZus{}value}\PY{p}{,} \PY{l+s+s1}{\PYZsq{}}\PY{l+s+s1}{.2f}\PY{l+s+s1}{\PYZsq{}}\PY{p}{)}
                    \PY{n}{trip\PYZus{}time}\PY{o}{.}\PY{n}{append}\PY{p}{(}\PY{n}{getvalue}\PY{p}{)}
                    
            \PY{k}{return} \PY{n}{trip\PYZus{}time}
\end{Verbatim}


    \begin{Verbatim}[commandchars=\\\{\}]
{\color{incolor}In [{\color{incolor}6}]:} \PY{c+c1}{\PYZsh{}\PYZsh{} Use this and additional cells to collect all of the trip times as a list \PYZsh{}\PYZsh{}}
        \PY{c+c1}{\PYZsh{}\PYZsh{} and then use pyplot functions to generate a histogram of trip times.     \PYZsh{}\PYZsh{}}
        
        \PY{c+c1}{\PYZsh{} load library}
        \PY{k+kn}{import} \PY{n+nn}{matplotlib}\PY{n+nn}{.}\PY{n+nn}{pyplot} \PY{k}{as} \PY{n+nn}{plt}
        
        \PY{c+c1}{\PYZsh{} this is a \PYZsq{}magic word\PYZsq{} that allows for plots to be displayed}
        \PY{c+c1}{\PYZsh{} inline with the notebook. If you want to know more, see:}
        \PY{c+c1}{\PYZsh{} http://ipython.readthedocs.io/en/stable/interactive/magics.html}
        \PY{o}{\PYZpc{}}\PY{k}{matplotlib} inline 
        
        \PY{c+c1}{\PYZsh{} as mentioned in let\PYZsq{}s choose any one city i.e. NYC}
        \PY{n}{data\PYZus{}file} \PY{o}{=} \PY{p}{\PYZob{}}\PY{l+s+s1}{\PYZsq{}}\PY{l+s+s1}{NYC}\PY{l+s+s1}{\PYZsq{}}\PY{p}{:} \PY{l+s+s1}{\PYZsq{}}\PY{l+s+s1}{./data/NYC\PYZhy{}2016\PYZhy{}Summary.csv}\PY{l+s+s1}{\PYZsq{}}\PY{p}{\PYZcb{}}
        
        \PY{c+c1}{\PYZsh{} example histogram, data taken from bay area sample}
        \PY{n}{data} \PY{o}{=} \PY{n}{collect\PYZus{}trip\PYZus{}time}\PY{p}{(}\PY{n}{data\PYZus{}file}\PY{p}{[}\PY{l+s+s1}{\PYZsq{}}\PY{l+s+s1}{NYC}\PY{l+s+s1}{\PYZsq{}}\PY{p}{]}\PY{p}{)}
        \PY{n}{plt}\PY{o}{.}\PY{n}{hist}\PY{p}{(}\PY{n}{data}\PY{p}{)}
        \PY{n}{plt}\PY{o}{.}\PY{n}{title}\PY{p}{(}\PY{l+s+s1}{\PYZsq{}}\PY{l+s+s1}{Distribution of Trip Durations}\PY{l+s+s1}{\PYZsq{}}\PY{p}{)}
        \PY{n}{plt}\PY{o}{.}\PY{n}{xlabel}\PY{p}{(}\PY{l+s+s1}{\PYZsq{}}\PY{l+s+s1}{Duration (m)}\PY{l+s+s1}{\PYZsq{}}\PY{p}{)}
        \PY{n}{plt}\PY{o}{.}\PY{n}{show}\PY{p}{(}\PY{p}{)}
\end{Verbatim}


    \begin{center}
    \adjustimage{max size={0.9\linewidth}{0.9\paperheight}}{output_24_0.png}
    \end{center}
    { \hspace*{\fill} \\}
    
    If you followed the use of the \texttt{.hist()} and \texttt{.show()}
functions exactly like in the example, you're probably looking at a plot
that's completely unexpected. The plot consists of one extremely tall
bar on the left, maybe a very short second bar, and a whole lot of empty
space in the center and right. Take a look at the duration values on the
x-axis. This suggests that there are some highly infrequent outliers in
the data. Instead of reprocessing the data, you will use additional
parameters with the \texttt{.hist()} function to limit the range of data
that is plotted. Documentation for the function can be found
\href{https://matplotlib.org/devdocs/api/_as_gen/matplotlib.pyplot.hist.html\#matplotlib.pyplot.hist}{{[}here{]}}.

\textbf{Question 5}: Use the parameters of the \texttt{.hist()} function
to plot the distribution of trip times for the Subscribers in your
selected city. Do the same thing for only the Customers. Add limits to
the plots so that only trips of duration less than 75 minutes are
plotted. As a bonus, set the plots up so that bars are in five-minute
wide intervals. For each group, where is the peak of each distribution?
How would you describe the shape of each distribution?

\textbf{Answer}: In case of \textbf{Subscribers}, duration between
\textbf{5-10} having peak distribution. In case of \textbf{Customers},
duration between \textbf{20-25} having peak distribution.

    \begin{Verbatim}[commandchars=\\\{\}]
{\color{incolor}In [{\color{incolor}7}]:} \PY{k}{def} \PY{n+nf}{collect\PYZus{}trip\PYZus{}time\PYZus{}modified}\PY{p}{(}\PY{n}{filename}\PY{p}{)}\PY{p}{:}
            \PY{l+s+sd}{\PYZdq{}\PYZdq{}\PYZdq{}}
        \PY{l+s+sd}{    functions to list  of the trip times less than 75 minutes for the city that is selected in question 4c. (i.e NYC)}
        \PY{l+s+sd}{    \PYZdq{}\PYZdq{}\PYZdq{}}
            \PY{k}{with} \PY{n+nb}{open}\PY{p}{(}\PY{n}{filename}\PY{p}{,}\PY{l+s+s1}{\PYZsq{}}\PY{l+s+s1}{r}\PY{l+s+s1}{\PYZsq{}}\PY{p}{)} \PY{k}{as} \PY{n}{f\PYZus{}in}\PY{p}{:}
                
                \PY{n}{reader} \PY{o}{=} \PY{n}{csv}\PY{o}{.}\PY{n}{DictReader}\PY{p}{(}\PY{n}{f\PYZus{}in}\PY{p}{)}
                \PY{n}{subscriber\PYZus{}trip\PYZus{}time} \PY{o}{=} \PY{p}{[}\PY{p}{]}
                \PY{n}{customer\PYZus{}trip\PYZus{}time} \PY{o}{=} \PY{p}{[}\PY{p}{]}
                
                \PY{k}{for} \PY{n}{row} \PY{o+ow}{in} \PY{n}{reader}\PY{p}{:}
                    \PY{n}{get\PYZus{}value} \PY{o}{=} \PY{n+nb}{float}\PY{p}{(}\PY{n}{row}\PY{p}{[}\PY{l+s+s1}{\PYZsq{}}\PY{l+s+s1}{duration}\PY{l+s+s1}{\PYZsq{}}\PY{p}{]}\PY{p}{)}
                    \PY{n}{getvalue} \PY{o}{=} \PY{n+nb}{int}\PY{p}{(}\PY{n}{get\PYZus{}value}\PY{p}{)}
                    \PY{k}{if}\PY{p}{(}\PY{n+nb}{float}\PY{p}{(}\PY{n}{getvalue}\PY{p}{)} \PY{o}{\PYZlt{}} \PY{l+m+mi}{75}\PY{p}{)}\PY{p}{:}
                        \PY{k}{if} \PY{p}{(}\PY{n}{row}\PY{p}{[}\PY{l+s+s1}{\PYZsq{}}\PY{l+s+s1}{user\PYZus{}type}\PY{l+s+s1}{\PYZsq{}}\PY{p}{]} \PY{o}{==} \PY{l+s+s1}{\PYZsq{}}\PY{l+s+s1}{Subscriber}\PY{l+s+s1}{\PYZsq{}}\PY{p}{)}\PY{p}{:}
                            \PY{n}{subscriber\PYZus{}trip\PYZus{}time}\PY{o}{.}\PY{n}{append}\PY{p}{(}\PY{p}{(}\PY{n}{getvalue}\PY{p}{)}\PY{p}{)}
                        \PY{k}{if} \PY{p}{(}\PY{n}{row}\PY{p}{[}\PY{l+s+s1}{\PYZsq{}}\PY{l+s+s1}{user\PYZus{}type}\PY{l+s+s1}{\PYZsq{}}\PY{p}{]} \PY{o}{==} \PY{l+s+s1}{\PYZsq{}}\PY{l+s+s1}{Customer}\PY{l+s+s1}{\PYZsq{}}\PY{p}{)}\PY{p}{:}
                            \PY{n}{customer\PYZus{}trip\PYZus{}time}\PY{o}{.}\PY{n}{append}\PY{p}{(}\PY{p}{(}\PY{n}{getvalue}\PY{p}{)}\PY{p}{)}
                    
            \PY{k}{return} \PY{n}{subscriber\PYZus{}trip\PYZus{}time}\PY{p}{,} \PY{n}{customer\PYZus{}trip\PYZus{}time}
\end{Verbatim}


    \begin{Verbatim}[commandchars=\\\{\}]
{\color{incolor}In [{\color{incolor}8}]:} \PY{c+c1}{\PYZsh{}\PYZsh{} Use this and additional cells to answer Question 5. \PYZsh{}\PYZsh{}}
        \PY{k+kn}{import} \PY{n+nn}{matplotlib}\PY{n+nn}{.}\PY{n+nn}{pyplot} \PY{k}{as} \PY{n+nn}{plt}
        
        \PY{o}{\PYZpc{}}\PY{k}{matplotlib} inline 
        
        \PY{c+c1}{\PYZsh{} as mentioned in let\PYZsq{}s choose any one city i.e. NYC}
        \PY{n}{data\PYZus{}file} \PY{o}{=} \PY{p}{\PYZob{}}\PY{l+s+s1}{\PYZsq{}}\PY{l+s+s1}{NYC}\PY{l+s+s1}{\PYZsq{}}\PY{p}{:} \PY{l+s+s1}{\PYZsq{}}\PY{l+s+s1}{./data/NYC\PYZhy{}2016\PYZhy{}Summary.csv}\PY{l+s+s1}{\PYZsq{}}\PY{p}{\PYZcb{}}
        
        \PY{c+c1}{\PYZsh{} example histogram, data taken from bay area sample}
        \PY{n}{subscriber\PYZus{}trip\PYZus{}time}\PY{p}{,} \PY{n}{customer\PYZus{}trip\PYZus{}time} \PY{o}{=} \PY{n}{collect\PYZus{}trip\PYZus{}time\PYZus{}modified}\PY{p}{(}\PY{n}{data\PYZus{}file}\PY{p}{[}\PY{l+s+s1}{\PYZsq{}}\PY{l+s+s1}{NYC}\PY{l+s+s1}{\PYZsq{}}\PY{p}{]}\PY{p}{)}
        \PY{c+c1}{\PYZsh{}ans = (max(subscriber\PYZus{}trip\PYZus{}time) \PYZhy{} min(subscriber\PYZus{}trip\PYZus{}time))}
        \PY{n}{intervals} \PY{o}{=} \PY{p}{[}\PY{l+m+mi}{0}\PY{p}{,}\PY{l+m+mi}{5}\PY{p}{,}\PY{l+m+mi}{10}\PY{p}{,}\PY{l+m+mi}{15}\PY{p}{,}\PY{l+m+mi}{20}\PY{p}{,}\PY{l+m+mi}{25}\PY{p}{,}\PY{l+m+mi}{30}\PY{p}{,}\PY{l+m+mi}{35}\PY{p}{,}\PY{l+m+mi}{40}\PY{p}{,}\PY{l+m+mi}{45}\PY{p}{,}\PY{l+m+mi}{50}\PY{p}{,}\PY{l+m+mi}{55}\PY{p}{,}\PY{l+m+mi}{60}\PY{p}{]}
        \PY{n}{plt}\PY{o}{.}\PY{n}{hist}\PY{p}{(}\PY{n}{subscriber\PYZus{}trip\PYZus{}time}\PY{p}{,} \PY{n}{bins} \PY{o}{=} \PY{n}{intervals}\PY{p}{)}
        \PY{n}{plt}\PY{o}{.}\PY{n}{title}\PY{p}{(}\PY{l+s+s2}{\PYZdq{}}\PY{l+s+s2}{Trip times for Subscribers}\PY{l+s+s2}{\PYZdq{}}\PY{p}{)} 
        \PY{n}{plt}\PY{o}{.}\PY{n}{xlabel}\PY{p}{(}\PY{l+s+s1}{\PYZsq{}}\PY{l+s+s1}{Duration}\PY{l+s+s1}{\PYZsq{}}\PY{p}{)}
        \PY{n}{plt}\PY{o}{.}\PY{n}{xticks}\PY{p}{(}\PY{n}{intervals}\PY{p}{)}
        \PY{n}{plt}\PY{o}{.}\PY{n}{show}\PY{p}{(}\PY{p}{)}
\end{Verbatim}


    \begin{center}
    \adjustimage{max size={0.9\linewidth}{0.9\paperheight}}{output_27_0.png}
    \end{center}
    { \hspace*{\fill} \\}
    
    \begin{Verbatim}[commandchars=\\\{\}]
{\color{incolor}In [{\color{incolor}9}]:} \PY{n}{intervals} \PY{o}{=} \PY{p}{[}\PY{l+m+mi}{0}\PY{p}{,}\PY{l+m+mi}{5}\PY{p}{,}\PY{l+m+mi}{10}\PY{p}{,}\PY{l+m+mi}{15}\PY{p}{,}\PY{l+m+mi}{20}\PY{p}{,}\PY{l+m+mi}{25}\PY{p}{,}\PY{l+m+mi}{30}\PY{p}{,}\PY{l+m+mi}{35}\PY{p}{,}\PY{l+m+mi}{40}\PY{p}{,}\PY{l+m+mi}{45}\PY{p}{,}\PY{l+m+mi}{50}\PY{p}{,}\PY{l+m+mi}{55}\PY{p}{,}\PY{l+m+mi}{60}\PY{p}{]}
        \PY{n}{plt}\PY{o}{.}\PY{n}{hist}\PY{p}{(}\PY{n}{customer\PYZus{}trip\PYZus{}time}\PY{p}{,} \PY{n}{bins} \PY{o}{=} \PY{n}{intervals}\PY{p}{)}
        \PY{n}{plt}\PY{o}{.}\PY{n}{title}\PY{p}{(}\PY{l+s+s2}{\PYZdq{}}\PY{l+s+s2}{Trip times for Customers}\PY{l+s+s2}{\PYZdq{}}\PY{p}{)} 
        \PY{n}{plt}\PY{o}{.}\PY{n}{xlabel}\PY{p}{(}\PY{l+s+s1}{\PYZsq{}}\PY{l+s+s1}{Duration}\PY{l+s+s1}{\PYZsq{}}\PY{p}{)}
        \PY{n}{plt}\PY{o}{.}\PY{n}{xticks}\PY{p}{(}\PY{n}{intervals}\PY{p}{)}
        \PY{n}{plt}\PY{o}{.}\PY{n}{show}\PY{p}{(}\PY{p}{)}
\end{Verbatim}


    \begin{center}
    \adjustimage{max size={0.9\linewidth}{0.9\paperheight}}{output_28_0.png}
    \end{center}
    { \hspace*{\fill} \\}
    
     \#\# Performing Your Own Analysis

So far, you've performed an initial exploration into the data available.
You have compared the relative volume of trips made between three U.S.
cities and the ratio of trips made by Subscribers and Customers. For one
of these cities, you have investigated differences between Subscribers
and Customers in terms of how long a typical trip lasts. Now it is your
turn to continue the exploration in a direction that you choose. Here
are a few suggestions for questions to explore:

\begin{itemize}
\tightlist
\item
  How does ridership differ by month or season? Which month / season has
  the highest ridership? Does the ratio of Subscriber trips to Customer
  trips change depending on the month or season?
\item
  Is the pattern of ridership different on the weekends versus weekdays?
  On what days are Subscribers most likely to use the system? What about
  Customers? Does the average duration of rides change depending on the
  day of the week?
\item
  During what time of day is the system used the most? Is there a
  difference in usage patterns for Subscribers and Customers?
\end{itemize}

If any of the questions you posed in your answer to question 1 align
with the bullet points above, this is a good opportunity to investigate
one of them. As part of your investigation, you will need to create a
visualization. If you want to create something other than a histogram,
then you might want to consult the
\href{https://matplotlib.org/devdocs/api/pyplot_summary.html}{Pyplot
documentation}. In particular, if you are plotting values across a
categorical variable (e.g.~city, user type), a bar chart will be useful.
The
\href{https://matplotlib.org/devdocs/api/_as_gen/matplotlib.pyplot.bar.html\#matplotlib.pyplot.bar}{documentation
page for \texttt{.bar()}} includes links at the bottom of the page with
examples for you to build off of for your own use.

\textbf{Question 6}: Continue the investigation by exploring another
question that could be answered by the data available. Document the
question you want to explore below. Your investigation should involve at
least two variables and should compare at least two groups. You should
also use at least one visualization as part of your explorations.

\textbf{Answer}: After performing analysis on months, I can see that
\textbf{between Nov-Dec} people used service most. Moreover, specific to
each user, \textbf{Subscriber} used most \textbf{between Nov-Dec} and
\textbf{Customers} used most \textbf{between Sept-Oct}.

    \begin{Verbatim}[commandchars=\\\{\}]
{\color{incolor}In [{\color{incolor}10}]:} \PY{k}{def} \PY{n+nf}{get\PYZus{}month\PYZus{}list}\PY{p}{(}\PY{n}{filename}\PY{p}{)}\PY{p}{:}
             \PY{l+s+sd}{\PYZdq{}\PYZdq{}\PYZdq{}}
         \PY{l+s+sd}{    functions which will return months list data the city NYC}
         \PY{l+s+sd}{    \PYZdq{}\PYZdq{}\PYZdq{}}
             \PY{k}{with} \PY{n+nb}{open}\PY{p}{(}\PY{n}{filename}\PY{p}{,}\PY{l+s+s1}{\PYZsq{}}\PY{l+s+s1}{r}\PY{l+s+s1}{\PYZsq{}}\PY{p}{)} \PY{k}{as} \PY{n}{f\PYZus{}in}\PY{p}{:}
                 
                 \PY{n}{reader} \PY{o}{=} \PY{n}{csv}\PY{o}{.}\PY{n}{DictReader}\PY{p}{(}\PY{n}{f\PYZus{}in}\PY{p}{)}
                 \PY{n}{tot\PYZus{}month\PYZus{}list} \PY{o}{=} \PY{p}{[}\PY{p}{]}
                 \PY{n}{subscribers\PYZus{}month\PYZus{}list} \PY{o}{=} \PY{p}{[}\PY{p}{]}
                 \PY{n}{customers\PYZus{}month\PYZus{}list} \PY{o}{=} \PY{p}{[}\PY{p}{]}
                 
                 \PY{k}{for} \PY{n}{row} \PY{o+ow}{in} \PY{n}{reader}\PY{p}{:}
                     \PY{n}{getvalue} \PY{o}{=} \PY{n+nb}{int}\PY{p}{(}\PY{n}{row}\PY{p}{[}\PY{l+s+s1}{\PYZsq{}}\PY{l+s+s1}{month}\PY{l+s+s1}{\PYZsq{}}\PY{p}{]}\PY{p}{)}
                     \PY{n}{tot\PYZus{}month\PYZus{}list}\PY{o}{.}\PY{n}{append}\PY{p}{(}\PY{n}{getvalue}\PY{p}{)}
                     \PY{k}{if} \PY{p}{(}\PY{n}{row}\PY{p}{[}\PY{l+s+s1}{\PYZsq{}}\PY{l+s+s1}{user\PYZus{}type}\PY{l+s+s1}{\PYZsq{}}\PY{p}{]} \PY{o}{==} \PY{l+s+s1}{\PYZsq{}}\PY{l+s+s1}{Subscriber}\PY{l+s+s1}{\PYZsq{}}\PY{p}{)}\PY{p}{:}
                         \PY{n}{subscribers\PYZus{}month\PYZus{}list}\PY{o}{.}\PY{n}{append}\PY{p}{(}\PY{n}{getvalue}\PY{p}{)}
                     \PY{k}{if} \PY{p}{(}\PY{n}{row}\PY{p}{[}\PY{l+s+s1}{\PYZsq{}}\PY{l+s+s1}{user\PYZus{}type}\PY{l+s+s1}{\PYZsq{}}\PY{p}{]} \PY{o}{==} \PY{l+s+s1}{\PYZsq{}}\PY{l+s+s1}{Customer}\PY{l+s+s1}{\PYZsq{}}\PY{p}{)}\PY{p}{:}
                         \PY{n}{customers\PYZus{}month\PYZus{}list}\PY{o}{.}\PY{n}{append}\PY{p}{(}\PY{n}{getvalue}\PY{p}{)}
             
             \PY{k}{return} \PY{n}{tot\PYZus{}month\PYZus{}list}\PY{p}{,} \PY{n}{subscribers\PYZus{}month\PYZus{}list}\PY{p}{,} \PY{n}{customers\PYZus{}month\PYZus{}list}
\end{Verbatim}


    \begin{Verbatim}[commandchars=\\\{\}]
{\color{incolor}In [{\color{incolor}11}]:} \PY{c+c1}{\PYZsh{}\PYZsh{} Which month has the highest ridership?                       \PYZsh{}\PYZsh{}}
         \PY{c+c1}{\PYZsh{}\PYZsh{} Does the ratio of Subscriber trips to Customer trips change  \PYZsh{}\PYZsh{}}
         \PY{c+c1}{\PYZsh{}\PYZsh{} depending on the month?                                      \PYZsh{}\PYZsh{}}
         
         \PY{c+c1}{\PYZsh{}\PYZsh{} Use this and additional cells to answer Question 5. \PYZsh{}\PYZsh{}}
         \PY{k+kn}{import} \PY{n+nn}{matplotlib}\PY{n+nn}{.}\PY{n+nn}{pyplot} \PY{k}{as} \PY{n+nn}{plt}
         \PY{k+kn}{import} \PY{n+nn}{csv}
         \PY{o}{\PYZpc{}}\PY{k}{matplotlib} inline 
         
         \PY{c+c1}{\PYZsh{} as mentioned in let\PYZsq{}s choose any one city i.e. NYC}
         \PY{n}{data\PYZus{}file} \PY{o}{=} \PY{p}{\PYZob{}}\PY{l+s+s1}{\PYZsq{}}\PY{l+s+s1}{NYC}\PY{l+s+s1}{\PYZsq{}}\PY{p}{:} \PY{l+s+s1}{\PYZsq{}}\PY{l+s+s1}{./data/NYC\PYZhy{}2016\PYZhy{}Summary.csv}\PY{l+s+s1}{\PYZsq{}}\PY{p}{\PYZcb{}}
         
         \PY{n}{tot\PYZus{}month\PYZus{}list}\PY{p}{,} \PY{n}{subscribers\PYZus{}month\PYZus{}list}\PY{p}{,} \PY{n}{customers\PYZus{}month\PYZus{}list} \PY{o}{=} \PY{n}{get\PYZus{}month\PYZus{}list}\PY{p}{(}\PY{n}{data\PYZus{}file}\PY{p}{[}\PY{l+s+s1}{\PYZsq{}}\PY{l+s+s1}{NYC}\PY{l+s+s1}{\PYZsq{}}\PY{p}{]}\PY{p}{)}
         \PY{n}{intervals} \PY{o}{=} \PY{p}{[}\PY{l+m+mi}{1}\PY{p}{,}\PY{l+m+mi}{2}\PY{p}{,}\PY{l+m+mi}{3}\PY{p}{,}\PY{l+m+mi}{4}\PY{p}{,}\PY{l+m+mi}{5}\PY{p}{,}\PY{l+m+mi}{6}\PY{p}{,}\PY{l+m+mi}{7}\PY{p}{,}\PY{l+m+mi}{8}\PY{p}{,}\PY{l+m+mi}{9}\PY{p}{,}\PY{l+m+mi}{10}\PY{p}{,}\PY{l+m+mi}{11}\PY{p}{,}\PY{l+m+mi}{12}\PY{p}{]}
         
         \PY{n}{plt}\PY{o}{.}\PY{n}{hist}\PY{p}{(}\PY{n}{tot\PYZus{}month\PYZus{}list}\PY{p}{,} \PY{n}{bins} \PY{o}{=} \PY{n}{intervals}\PY{p}{)}
         \PY{n}{plt}\PY{o}{.}\PY{n}{title}\PY{p}{(}\PY{l+s+s2}{\PYZdq{}}\PY{l+s+s2}{tot\PYZus{}month\PYZus{}list}\PY{l+s+s2}{\PYZdq{}}\PY{p}{)} 
         \PY{n}{plt}\PY{o}{.}\PY{n}{xlabel}\PY{p}{(}\PY{l+s+s1}{\PYZsq{}}\PY{l+s+s1}{Month Number}\PY{l+s+s1}{\PYZsq{}}\PY{p}{)}
         \PY{n}{plt}\PY{o}{.}\PY{n}{xticks}\PY{p}{(}\PY{n}{intervals}\PY{p}{)}
         \PY{n}{plt}\PY{o}{.}\PY{n}{show}\PY{p}{(}\PY{p}{)}
\end{Verbatim}


    \begin{center}
    \adjustimage{max size={0.9\linewidth}{0.9\paperheight}}{output_31_0.png}
    \end{center}
    { \hspace*{\fill} \\}
    
    \begin{Verbatim}[commandchars=\\\{\}]
{\color{incolor}In [{\color{incolor}12}]:} \PY{n}{plt}\PY{o}{.}\PY{n}{hist}\PY{p}{(}\PY{n}{subscribers\PYZus{}month\PYZus{}list}\PY{p}{,} \PY{n}{bins} \PY{o}{=} \PY{n}{intervals}\PY{p}{)}
         \PY{n}{plt}\PY{o}{.}\PY{n}{title}\PY{p}{(}\PY{l+s+s2}{\PYZdq{}}\PY{l+s+s2}{subscribers\PYZus{}month\PYZus{}list}\PY{l+s+s2}{\PYZdq{}}\PY{p}{)}
         \PY{n}{plt}\PY{o}{.}\PY{n}{xlabel}\PY{p}{(}\PY{l+s+s1}{\PYZsq{}}\PY{l+s+s1}{Month Number}\PY{l+s+s1}{\PYZsq{}}\PY{p}{)}
         \PY{n}{plt}\PY{o}{.}\PY{n}{xticks}\PY{p}{(}\PY{n}{intervals}\PY{p}{)}
         \PY{n}{plt}\PY{o}{.}\PY{n}{show}\PY{p}{(}\PY{p}{)}
\end{Verbatim}


    \begin{center}
    \adjustimage{max size={0.9\linewidth}{0.9\paperheight}}{output_32_0.png}
    \end{center}
    { \hspace*{\fill} \\}
    
    \begin{Verbatim}[commandchars=\\\{\}]
{\color{incolor}In [{\color{incolor}13}]:} \PY{n}{plt}\PY{o}{.}\PY{n}{hist}\PY{p}{(}\PY{n}{customers\PYZus{}month\PYZus{}list}\PY{p}{,} \PY{n}{bins} \PY{o}{=} \PY{n}{intervals}\PY{p}{)}
         \PY{n}{plt}\PY{o}{.}\PY{n}{title}\PY{p}{(}\PY{l+s+s2}{\PYZdq{}}\PY{l+s+s2}{customers\PYZus{}month\PYZus{}list}\PY{l+s+s2}{\PYZdq{}}\PY{p}{)} 
         \PY{n}{plt}\PY{o}{.}\PY{n}{xlabel}\PY{p}{(}\PY{l+s+s1}{\PYZsq{}}\PY{l+s+s1}{Month Number}\PY{l+s+s1}{\PYZsq{}}\PY{p}{)}
         \PY{n}{plt}\PY{o}{.}\PY{n}{xticks}\PY{p}{(}\PY{n}{intervals}\PY{p}{)}
         \PY{n}{plt}\PY{o}{.}\PY{n}{show}\PY{p}{(}\PY{p}{)}
\end{Verbatim}


    \begin{center}
    \adjustimage{max size={0.9\linewidth}{0.9\paperheight}}{output_33_0.png}
    \end{center}
    { \hspace*{\fill} \\}
    
     \#\# Conclusions

Congratulations on completing the project! This is only a sampling of
the data analysis process: from generating questions, wrangling the
data, and to exploring the data. Normally, at this point in the data
analysis process, you might want to draw conclusions about the data by
performing a statistical test or fitting the data to a model for making
predictions. There are also a lot of potential analyses that could be
performed on the data which are not possible with only the data
provided. For example, detailed location data has not been investigated.
Where are the most commonly used docks? What are the most common routes?
As another example, weather has potential to have a large impact on
daily ridership. How much is ridership impacted when there is rain or
snow? Are subscribers or customers affected more by changes in weather?

\textbf{Question 7}: Putting the bike share data aside, think of a topic
or field of interest where you would like to be able to apply the
techniques of data science. What would you like to be able to learn from
your chosen subject?

\textbf{Answer}: From large amount of record, we can analyze that how
many Android developers, IOS developers, web techenology developers,
Machine learning developers or other develpers are available. From this
analysis we can predict that, in which field there is very less
develpers and which developer will in demand most in upcoming future.

\begin{quote}
\textbf{Tip}: If we want to share the results of our analysis with
others, we aren't limited to giving them a copy of the jupyter Notebook
(.ipynb) file. We can also export the Notebook output in a form that can
be opened even for those without Python installed. From the
\textbf{File} menu in the upper left, go to the \textbf{Download as}
submenu. You can then choose a different format that can be viewed more
generally, such as HTML (.html) or PDF (.pdf). You may need additional
packages or software to perform these exports.
\end{quote}

\begin{quote}
If you are working on this project via the Project Notebook page in the
classroom, you can also submit this project directly from the workspace.
\textbf{Before you do that}, you should save an HTML copy of the
completed project to the workspace by running the code cell below. If it
worked correctly, the output code should be a 0, and if you click on the
jupyter icon in the upper left, you should see your .html document in
the workspace directory. Alternatively, you can download the .html copy
of your report following the steps in the previous paragraph, then
\emph{upload} the report to the directory (by clicking the jupyter
icon).
\end{quote}

\begin{quote}
Either way, once you've gotten the .html report in your workspace, you
can complete your submission by clicking on the ``Submit Project''
button to the lower-right hand side of the workspace.
\end{quote}

    \begin{Verbatim}[commandchars=\\\{\}]
{\color{incolor}In [{\color{incolor} }]:} \PY{k+kn}{from} \PY{n+nn}{subprocess} \PY{k}{import} \PY{n}{call}
        \PY{n}{call}\PY{p}{(}\PY{p}{[}\PY{l+s+s1}{\PYZsq{}}\PY{l+s+s1}{python}\PY{l+s+s1}{\PYZsq{}}\PY{p}{,} \PY{l+s+s1}{\PYZsq{}}\PY{l+s+s1}{\PYZhy{}m}\PY{l+s+s1}{\PYZsq{}}\PY{p}{,} \PY{l+s+s1}{\PYZsq{}}\PY{l+s+s1}{nbconvert}\PY{l+s+s1}{\PYZsq{}}\PY{p}{,} \PY{l+s+s1}{\PYZsq{}}\PY{l+s+s1}{Bike\PYZus{}Share\PYZus{}Analysis.ipynb}\PY{l+s+s1}{\PYZsq{}}\PY{p}{]}\PY{p}{)}
\end{Verbatim}



    % Add a bibliography block to the postdoc
    
    
    
    \end{document}
